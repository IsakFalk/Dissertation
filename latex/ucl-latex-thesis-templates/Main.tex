% UCL Thesis LaTeX Template
%  (c) Ian Kirker, 2014
% 
% This is a template/skeleton for PhD/MPhil/MRes theses.
%
% It uses a rather split-up file structure because this tends to
%  work well for large, complex documents.
% We suggest using one file per chapter, but you may wish to use more
%  or fewer separate files than that.
% We've also separated out various bits of configuration into their
%  own files, to keep everything neat.
% Note that the \input command just streams in whatever file you give
%  it, while the \include command adds a page break, and does some
%  extra organisation to make compilation faster. Note that you can't
%  use \include inside an \include-d file.
% We suggest using \input for settings and configuration files that
%  you always want to use, and \include for each section of content.
% If you do that, it also means you can use the \includeonly statement
%  to only compile up the section you're currently interested in.
% You might also want to put figures into their own files to be \input.

% For more information on \input and \include, see:
%  http://tex.stackexchange.com/questions/246/when-should-i-use-input-vs-include


% Formatting rules for theses are here: 
%  http://www.ucl.ac.uk/current-students/research_degrees/thesis_formatting
% Binding and submitting guidelines are here:
%  http://www.ucl.ac.uk/current-students/research_degrees/thesis_binding_submission

% This package goes first and foremost, because it checks all 
%  your syntax for mistakes and some old-fashioned LaTeX commands.
% Note that normally you should load your documentclass before 
%  packages, because some packages change behaviour based on
%  your document settings.
% Also, for those confused by the RequirePackage here vs usepackage
%  elsewhere, usepackage cannot be used before the documentclass
%  command, while RequirePackage can. That's the only functional
%  difference as far as I'm aware.
\RequirePackage[l2tabu, orthodox]{nag}

% ------ Main document class specification ------
% The draft option here prevents images being inserted,
%  and adds chunky black bars to boxes that are exceeding
%  the page width (to show that they are). 
% The oneside option can optionally be replaced by twoside if
%  you intend to print double-sided. Note that this is
%  *specifically permitted* by the UCL thesis formatting
%  guidelines.
%
% Valid options in terms of type are:
%  phd
%  mres
%  mphil 
%  \documentclass[12pt,phd,draft,a4paper,oneside]{ucl_thesis}
\documentclass[12pt,mphil,a4paper,oneside]{ucl_thesis}

% Include the .sty file. Check so that it doesn't fuck up
% the whole document. Might want to rewrite it.
\usepackage{Dissertation}
\makenomenclature

% Path to the graphics (figures, plots etc)
\graphicspath{{./figures}}
% Package configuration:
%  LaTeX uses "packages" to add extra commands and features.
%  There are quite a few useful ones, so we've put them in a 
%   separate file.
% -------- Packages --------

% This package just gives you a quick way to dump in some sample text.
% You can remove it -- it's just here for the examples.
\usepackage{blindtext}

% This package means empty pages (pages with no text) won't get stuff
%  like chapter names at the top of the page. It's mostly cosmetic.
\usepackage{emptypage}

% The graphicx package adds the \includegraphics command,
%  which is your basic command for adding a picture.
\usepackage{graphicx}

% The float package improves LaTeX's handling of floats,
%  and also adds the option to *force* LaTeX to put the float
%  HERE, with the [H] option to the float environment.
\usepackage{float}

% The amsmath package enhances the various ways of including
%  maths, including adding the align environment for aligned
%  equations.
\usepackage{amsmath}

% Use these two packages together -- they define symbols
%  for e.g. units that you can use in both text and math mode.
\usepackage{gensymb}
\usepackage{textcomp}
% You may also want the units package for making little
%  fractions for unit specifications.
%\usepackage{units}


% The setspace package lets you use 1.5-sized or double line spacing.
\usepackage{setspace}
\setstretch{1.5}

% That just does body text -- if you want to expand *everything*,
%  including footnotes and tables, use this instead:
%\renewcommand{\baselinestretch}{1.5}


% PGFPlots is either a really clunky or really good way to add graphs
%  into your document, depending on your point of view.
% There's waaaaay too much information on using this to cover here,
%  so, you might want to start here:
%   http://pgfplots.sourceforge.net/
%  or here:
%   http://pgfplots.sourceforge.net/pgfplots.pdf
%\usepackage{pgfplots}
%\pgfplotsset{compat=1.3} % <- this fixed axis labels in the version I was using

% PGFPlotsTable can help you make tables a little more easily than
%  usual in LaTeX.
% If you're going to have to paste data in a lot, I'd suggest using it.
%  You might want to start with the manual, here:
%  http://pgfplots.sourceforge.net/pgfplotstable.pdf
%\usepackage{pgfplotstable}

% These settings are also recommended for using with pgfplotstable.
%\pgfplotstableset{
%	% these columns/<colname>/.style={<options>} things define a style
%	% which applies to <colname> only.
%	empty cells with={--}, % replace empty cells with '--'
%	every head row/.style={before row=\toprule,after row=\midrule},
%	every last row/.style={after row=\bottomrule}
%}


% The mhchem package provides chemistry formula typesetting commands
%  e.g. \ce{H2O}
%\usepackage[version=3]{mhchem}

% And the chemfig package gives a weird command for adding Lewis 
%  diagrams, for e.g. organic molecules
%\usepackage{chemfig}

% The linenumbers command from the lineno package adds line numbers
%  alongside your text that can be useful for discussing edits 
%  in drafts.
% Remove or comment out the command for proper versions.
%\usepackage[modulo]{lineno}
% \linenumbers 


% Alternatively, you can use the ifdraft package to let you add
%  commands that will only be used in draft versions
%\usepackage{ifdraft}

% For example, the following adds a watermark if the draft mode is on.
%\ifdraft{
%  \usepackage{draftwatermark}
%  \SetWatermarkText{\shortstack{\textsc{Draft Mode}\\ \strut \\ \strut \\ \strut}}
%  \SetWatermarkScale{0.5}
%  \SetWatermarkAngle{90}
%}


% The multirow package adds the option to make cells span 
%  rows in tables.
\usepackage{multirow}


% Subfig allows you to create figures within figures, to, for example,
%  make a single figure with 4 individually labeled and referenceable
%  sub-figures.
% It's quite fiddly to use, so check the documentation.
%\usepackage{subfig}

% The natbib package allows book-type citations commonly used in
%  longer works, and less commonly in science articles (IME).
% e.g. (Saucer et al., 1993) rather than [1]
% More details are here: http://merkel.zoneo.net/Latex/natbib.php
%\usepackage{natbib}

% The bibentry package (along with the \nobibliography* command)
%  allows putting full reference lines inline.
%  See: 
%   http://tex.stackexchange.com/questions/2905/how-can-i-list-references-from-bibtex-file-in-line-with-commentary
\usepackage{bibentry} 

% The isorot package allows you to put things sideways 
%  (or indeed, at any angle) on a page.
% This can be useful for wide graphs or other figures.
%\usepackage{isorot}

% The caption package adds more options for caption formatting.
% This set-up makes hanging labels, makes the caption text smaller
%  than the body text, and makes the label bold.
% Highly recommended.
\usepackage[format=hang,font=small,labelfont=bf]{caption}

% If you're getting into defining your own commands, you might want
%  to check out the etoolbox package -- it defines a few commands
%  that can make it easier to make commands robust.
\usepackage{etoolbox}
 
% Sets up links within your document, for e.g. contents page entries
%  and references, and also PDF metadata.
% You should edit this!
%%
%% This file uses the hyperref package to make your thesis have metadata embedded in the PDF, 
%%  and also adds links to be able to click on references and contents page entries to go to 
%%  the pages.
%%

% Some hacks are necessary to make bibentry and hyperref play nicely.
% See: http://tex.stackexchange.com/questions/65348/clash-between-bibentry-and-hyperref-with-bibstyle-elsart-harv
\usepackage{bibentry}
\makeatletter\let\saved@bibitem\@bibitem\makeatother
\usepackage[pdftex,hidelinks]{hyperref}
\makeatletter\let\@bibitem\saved@bibitem\makeatother
\makeatletter
\AtBeginDocument{
    \hypersetup{
        pdfsubject={Thesis Subject},
        pdfkeywords={Thesis Keywords},
        pdfauthor={Author},
        pdftitle={Title},
    }
}
\makeatother
    


% And then some settings in separate files.
% These settings are from:
%  http://mintaka.sdsu.edu/GF/bibliog/latex/floats.html

% They give LaTeX more options on where to put your figures, and may
%  mean that fewer of your figures end up at the tops of pages far
%  away from the thing they're related to.

% Alters some LaTeX defaults for better treatment of figures:
% See p.105 of "TeX Unbound" for suggested values.
% See pp. 199-200 of Lamport's "LaTeX" book for details.

%   General parameters, for ALL pages:
\renewcommand{\topfraction}{0.9}	% max fraction of floats at top
\renewcommand{\bottomfraction}{0.8}	% max fraction of floats at bottom

%   Parameters for TEXT pages (not float pages):
\setcounter{topnumber}{2}
\setcounter{bottomnumber}{2}
\setcounter{totalnumber}{4}     % 2 may work better
\setcounter{dbltopnumber}{2}    % for 2-column pages
\renewcommand{\dbltopfraction}{0.9}	% fit big float above 2-col. text
\renewcommand{\textfraction}{0.07}	% allow minimal text w. figs

%   Parameters for FLOAT pages (not text pages):
\renewcommand{\floatpagefraction}{0.7}	% require fuller float pages
% N.B.: floatpagefraction MUST be less than topfraction !!
\renewcommand{\dblfloatpagefraction}{0.7}	% require fuller float pages

% remember to use [htp] or [htpb] for placement,
% e.g. 
%  \begin{figure}[htp]
%   ...
%  \end{figure} % For things like figures and tables
\bibliographystyle{unsrt}   % For bibliographies

% These control how many number sections your subsections will take
%    e.g. Section 2.3.1.5.6.3
%  and how many of those will get put into the contents pages.
\setcounter{secnumdepth}{3}
\setcounter{tocdepth}{3}

\begin{document}

\nobibliography*
% ^-- This is a dumb trick that works with the bibentry package to let
%  you put bibliography entries whereever you like.
% I used this to put references to papers a chapter's work was 
%  published in at the end of that chapter.
% For more information, see: http://stefaanlippens.net/bibentry

% At last, content! Remember filenames are case-sensitive and 
%  *must not* include spaces.
% I may change the way this is done in a future version, 
%  but given that some people needed it, if you need a different degree title 
%  (e.g. Master of Science, Master in Science, Master of Arts, etc)
%  uncomment the following 3 lines and set as appropriate (this *has* to be before \maketitle)
\makeatletter
\renewcommand {\@degree@string} {Master of Science}
\makeatother

\title{Latent Variable Models in Neural Machine Translation}
\author{John Isak Texas Falk}
\department{The Centre for Computational Statistics and Machine Learning}

\maketitle
%\makedeclaration

\leavevmode\thispagestyle{empty}\newpage

% Info in: https://users.ece.cmu.edu/~koopman/essays/abstract.html

\begin{abstract}
  Neural machine translation is the application of neural networks to perform
  translation between two languages, natural language generation concerns itself
  with the generation of sentences using a probabilistic model. Employing latent
  variable models for language generation enables generation of sentences from
  an underlying latent cause. Latent variable models using neural
  networks are very powerful but hard to optimise since the log-likelihood is
  intractable. Variational Autoencoder is a framework for optimising a stochastic approximation of the
  ELBO, a lower bound on the log-likelihood, letting us train these models. While variational autoencoders have been applied with some success to natural language
  processing, it is still unclear if this could be used as a way of translating
  sentence between two languages. We apply the variational autoencoder framework
  to a latent variable model mapping the latent variable to a sentence in two
  distinct languages, showing that after completing training, we can perform
  translation through the use of the recognition model. We evaluate how the
  quality of the generated and translated sentences differ with the power of the
  generative model by qualitative inspection, BLEU, upper bounded perplexity and
  an estimate of the ELBO on a small and big dataset. The results show that
  while the model manages to perform well on shorter sentences it has problems
  generating coherent sentences over longer sequences and that translation is
  possible but unlikely to be competitive with the currently best models.
\end{abstract}

\begin{acknowledgements}
I would like to thank the help from my supervisor Harshil Shah for helping me
make this thesis possible, and my parents, for always being there for me.
\end{acknowledgements}

\setcounter{tocdepth}{2} 
% Setting this higher means you get contents entries for
%  more minor section headers.

\tableofcontents
\listoffigures
\listoftables

% Acronyms
\nomenclature[A]{NLP}{Natural Language Processing}
\nomenclature[A]{NMT}{Neural Machine Translation}
\nomenclature[A]{MLE}{Maximum Likelihood Estimation}
\nomenclature[A]{MCMC}{Markov Chain Monte Carlo}
\nomenclature[A]{VI}{Variational Inference}
\nomenclature[A]{GD}{Gradient Descent}
\nomenclature[A]{SGD}{Stochastic Gradient Descent}
% Language
\nomenclature[L]{$\sigma(x)$}{An activation function with respect to $x$}
\nomenclature[L]{$\mathcal{M}$}{A model $\mathcal{M}$}
\nomenclature[L]{$\bm{\theta} \in \bm{\Theta}$}{Parameters $\bm{\theta}$
  belonging to the parameter space $\bm{\Theta}$}
\nomenclature[L]{$N$}{Number of data points}
\nomenclature[L]{$\mathcal{X}$}{Set of sentences belonging to language $X$}
\nomenclature[L]{$\mathcal{Y}$}{Set of sentences belonging to language $Y$}
\nomenclature[L]{$\bm{x}_i \in \mathcal{Y}$}{Sentence $\bm{x}_i$ belonging to
  language $X$}
\nomenclature[L]{$\bm{y}_i \in \mathcal{Y}$}{Sentence $\bm{y}_i$ belonging to
  language $Y$}
\nomenclature[L]{$\mathsf{w}$}{The word as a symbol}
\nomenclature[L]{$\bm{w}$}{One-hot encoding of $\mathsf{w}$}
\nomenclature[L]{$\bm{v}_{\mathsf{w}}$}{Word embedding of $\mathsf{w}$}
\nomenclature[L]{$V$}{Dictionary; set of all words in vocabulary}
% Mathematics
\nomenclature[M]{$\mathbb{R}$}{Set of real numbers}
\nomenclature[M]{$\mathbb{R}^d$}{Real vector space of dimension $d$}
\nomenclature[M]{$\bm{x}$}{A real column vector}
\nomenclature[M]{$\bm{I}_d$}{The $d \times d$ unit matrix}
\nomenclature[M]{$\bm{0}_d$}{The $d$-dimensional zero vector}
\nomenclature[M]{$\bm{0}_{d \times d}$}{The $d \times d$-dimensional zero matrix}
\nomenclature[M]{$\bm{A} \in \mathbf{R}^{n \times m}$}{Real matrix $\bm{A}$ of
  dimensions $n \times m$}
\nomenclature[M]{$\bm{A} \odot \bm{B}$}{Elementwise multiplication of matrices
  $\bm{A}$ and $\bm{B}$}
\nomenclature[M]{$\text{pa}(x)$}{Set of parent nodes for node $x$ in a
  graphical model}
\nomenclature[M]{$\nabla_{\bm{x}} Q(\bm{x})$}{Gradient of function $Q$ with respect to
  vector $\bm{x}$}
% Probability
\nomenclature[P]{$\mathcal{D}$}{Set of data points}
\nomenclature[P]{$p(x)$}{Probability distribution with respect to random
  variable $x$}
\nomenclature[P]{$\E_x[f(x, y)]$}{Expectation of a function $f(x, y)$ with
  respect to a random variable $x$. If no ambiguity to which
  variable we are taking expectation with respect to exists the subscript may be
  left out}
\nomenclature[P]{$\Cov(x, y)$}{Covariance with respect to $x$ and $y$}
\nomenclature[P]{$\Cov(\bm{x}, \bm{y})$}{Multivariate covariance with respect
  to $\bm{x}$ and $\bm{y}$}
\nomenclature[P]{$\mathcal{N}(\bm{\mu}, \bm{\Sigma})$}{Normal distribution
  with mean $\bm{\mu}$ and covariance matrix $\bm{\Sigma}$}
\nomenclature[P]{$\mathcal{N}(\bm{x} | \bm{\mu}, \bm{\Sigma})$}{$\bm{x}$
  belongs to a normal distribution with mean $\bm{\mu}$ and covariance matrix
  $\bm{\Sigma}$}
\nomenclature[P]{$\bm{x} \sim \mathcal{N}(\bm{x} | \bm{\mu}, \bm{\Sigma})$}{$\bm{x}$
  is distributed as a normal distribution with mean $\bm{\mu}$ and covariance matrix
  $\bm{\Sigma}$}
\nomenclature[P]{$\mathcal{L}(\bm{\theta} | \mathcal{D})$}{Likelihood
  function for $\bm{\theta}$ with respect to data $\mathcal{D}$}
\nomenclature[P]{$\ell(\bm{\theta} | \mathcal{D})$}{Log-likelihood
  function for $\bm{\theta}$ with respect to data $\mathcal{D}$}
\nomenclature[P]{$\KL{p, q}$}{Kullback-Leibler divergence from distribution
  $q$ to $p$}
\printnomenclature

\chapter{Introductory Material}
\label{chapterlabel1}

Some stuff about things.\cite{example-citation} Some more things. 

Inline citation: \bibentry{example-citation}

% This just dumps some pseudolatin in so you can see some text in place.
\blindtext

\chapter{Background Knowledge}
\label{BackgroundKnowledgeCh}

From here on I will assume familiarity with some concepts which will be
important for the experiments that we will conduct and analyse.

\section{Probability Theory and Statistics}

Probability theory is a natural candidate for doing principled reasoning about
uncertainty, in that is isomorphic to believes which follow Cox desiderata. This
is one of the reasons it has gained such widespread use in machine learning\cite[p.~3-23]{jaynes2003probability}.

\subsection{Rules and Theorems}
(might just want to dump Maneesh's stuff on probability basics, page 2)

Rules of probability that will be useful to us are the following, if we let $X
\in \mathcal{X}$ be a random variable and $p(X)$ the measure with respect to
this variable, then
\begin{description}
\item[Unit volume]
  \begin{equation}
    \label{eq:unit_vol_prob_axiom}
    \int_{\mathcal{X}}p(X) \dif X = 1
  \end{equation}
\item[Non-negativity]
  \begin{equation}
    \label{eq:non_neg_of_prob}
    p(X) \geq 0
  \end{equation}
\end{description}
the integral is interpreted as the Lebesgue integral if $X$ is continuous and as
a sum over the possible values of $X$ if it is discrete.

I will assume familiarity with random variables and random vectors, difference
between continuous and discrete variables.

Most manipulation of statements about random variables can be stated as a consequence
of the two following fundamental rules, given two random variables $X, Y$
defined on the support $\mathcal{X}, \mathcal{Y}$ respectively, then
\begin{description}
\item[Sum rule]
  \begin{equation}
    \label{eq:sum_rule}
    p(X) = \int_{\mathcal{Y}}p(X, Y) \dif Y
  \end{equation}
\item[Product rule]
  \begin{equation}
    \label{eq:product_rule}
    p(X, Y) = p(Y | X)p(X)
  \end{equation}
\end{description}

Two very important operations involving probabilities of random variables are
those of \textit{Expectation} and \textit{Covariance}. These take as input a
function $f$ and maps to the real number line $\mathbb{R}$, and are defined
implicitly with regards to some random variable $X$ and its 
probability distribution $p(X)$.
\begin{description}
\item[Expectation]
  \begin{equation}
    \label{eq:expectation}
    \E_X[f] = \int_{\mathcal{X}} f(X) p(X) \dif X
  \end{equation}
\item[Covariance]
  \begin{equation}
    \label{eq:covariance}
    \Cov(X, Y) = \E_{XY}[(X - \E_X[X])(Y - \E_Y[Y])]
  \end{equation}
\end{description}
We then define the variance operator as
\begin{equation}
  \label{eq:variance}
  \Var(X) = \Cov(X, X)
\end{equation}

The generalisation from $f: \mathcal{X} \to \mathbb{R}$ to $f: \mathcal{X} \to
\mathbb{R}^n$ is defined in the straightforward manner such that if $\bm{f} =
f(X)$ then
\begin{equation*}
  \E_X
  \begin{bmatrix}
    \bm{f}_1 \\
    \vdots \\
    \bm{f}_n \\
  \end{bmatrix} =
  \begin{bmatrix}
    \E_X \bm{f}_1 \\
    \vdots \\
    \E_X \bm{f}_n \\
  \end{bmatrix}
\end{equation*}
similarly $\Cov(\bm{f})$ is a $D \times D$-dimensional matrix where
$\Cov(\bm{f})_{i,j} = \Cov(\bm{f}_i, \bm{f}_j)$.\cite{Bishop:2006}

\subsection{The Gaussian Distribution}

For a $D$-dimensional random vector $\bm{X}$, the multivariate Gaussian
distribution takes the form
\begin{equation}
  \label{eq:Gaussian_dist}
  \mathcal{N}(\bm{X} | \bm{\mu}, \bm{\Sigma}) = \frac{1}{(2\pi)^{D/2}}\frac{1}{|\bm{\Sigma}|^{1/2}}\exp\left( -\frac{1}{2}(\bm{X} - \bm{\mu})^T\bm{\Sigma}^{-1}(\bm{X} - \bm{\mu})\right)
\end{equation}
where $\bm{\mu}$ is a $D$-dimensional mean vector, $\bm{\Sigma}$ is a $D \times
D$ dimensional positive definite covariance matrix, and $|\bm{\Sigma}|$ denotes
the determinant of $\bm{\Sigma}$. It is straightforward to show that these
parameters correspond to the mean and covariance as defined in equations
\eqref{eq:expectation} and \eqref{eq:covariance}\cite{}.

The Gaussian distribution can be seen as a unit $D$-dimensional cube which is
translated, sheared and rotated, giving rise to the fact that we can write any
Gaussianly distributed random variable $\bm{x} \sim \mathcal{N}(\bm{x} |
\bm{\mu}, \bm{\Sigma})$ as a linear combination of a unit Gaussian random
variable $\bm{z} \sim \mathcal{N}(\bm{x} | \bm{0}_D, \bm{I}_{D \times D})$. If
we let $\bm{\Lambda} \bm{\Lambda}^{\top} = \bm{\Sigma}$ be the Cholesky
decomposition\cite[p.~100-102]{Press:2007:NRE:1403886} of $\bm{\Sigma}$, then we
also have that
\begin{equation}
  \label{eq:sample_x}
  \bm{x} = \bm{\mu} + \bm{\Lambda}\bm{z}
\end{equation},
where the equality is in terms of distribution. If we further assume that
$\bm{x}$ is parametrised by $\bm{\mu}$ and $\bm{\Sigma}$ such that $\bm{\Sigma}$
is diagonal positive definite with diagonal $\bm{\sigma}$, then $\bm{\Sigma} =
\bm{\sigma} \odot \bm{I}_{D \times D}$. Finally this means that if we want to
sample a random variable $\bm{x}$ with diagonal covariance structure, then we
can do this by sampling a unit normal $\bm{z}$ which we then transform, which we can express as
\begin{equation}
  \label{eq:sample_x_diag_covariance}
  \bm{x} = \bm{\mu} + \bm{\sigma} \odot \bm{z} \sim \mathcal{N}(\bm{\mu}, \bm{\sigma} \odot \bm{I}_{D \times D})
\end{equation}

As the Gaussian distribution is part of the exponential family, the density of
joint distribution of iid Gaussian variables are themselves Gaussian distributed
where the natural parameters of this joint distribution is the sum of the
natural parameters of each random variable in the joint. In particular for the
Gaussian distribution, this means that if we have a collection of iid gaussian
random variables $\{\bm{x}_i)\}_i^n$, such that $\bm{x}_i \sim
\mathcal{N}(\bm{x}_i | \bm{\mu}_i, \bm{\Sigma}_{i})$, then the joint
can be found to be Gaussian distributed as
\begin{equation*}
  \mathcal{N}(\bm{\mu}, \bm{\Sigma})
\end{equation*},
where
\begin{align}
  \bm{\Sigma} & = \left( \sum_i^n \bm{\Sigma}_i^{-1} \right)^{-1} \label{eq:joint_indep_normal_covariance}\\ 
  \bm{\mu} & = \bm{\Sigma}\left( \sum_i^n \bm{\Sigma}^{-1} \bm{\mu}_i \right) \label{eq:joint_indep_normal_mean}
\end{align}\cite[p.~78-84]{Bishop:2006}.

In the case of two random variables distributed according to the form as laid
out in \eqref{eq:sample_x}, $\bm{x} \sim
\mathcal{N}(\mathcal{N}(\bm{\mu}_{\bm{x}}, \bm{\sigma}_{\bm{x}} \odot \bm{I}))$
and $\bm{y} \sim \mathcal{N}(\mathcal{N}(\bm{\mu}_{\bm{y}}, \bm{\sigma}_{\bm{y}}
\odot \bm{I}))$ we have that the resulting distribution $p(\bm{x}, \bm{y}) =
p(\bm{x})p(\bm{y})$ is distributed such that
\begin{equation*}
  p(\bm{x}, \bm{y}) = \mathcal{N}(\bm{\mu}_{\bm{x}, \bm{y}}, \bm{\sigma}_{\bm{x}, \bm{y}})
\end{equation*}
where
\begin{align}
  \bm{\sigma}_{\bm{x}, \bm{y}} & = \frac{1}{\bm{\sigma}_{\bm{x}}^{-1} + \bm{\sigma}_{\bm{y}}^{-1}} \label{eq:joint_indep_normal_covariance_diag}\\
  \bm{\mu}_{\bm{x}, \bm{y}} & = \frac{\bm{\sigma}_{\bm{x}}^{-1}\bm{\mu}_{\bm{x}} + \bm{\sigma}_{\bm{y}}^{-1}\bm{\mu}_{\bm{y}}}{\bm{\sigma}_{\bm{x}}^{-1} + \bm{\sigma}_{\bm{y}}^{-1}} \label{eq:joint_indep_normal_mean_diag}
\end{align}.

\subsection{Maximum Likelihood Estimation}

Assume we have a model $\mathcal{M}$ parametrised by $\bm{\theta}$ constrained
to live in the parameter space $\bm{\Theta}$. Given data $\mathcal{D}$ we want
to be able to fit the parameters $\bm{\theta}$ such that these generalise to
unseen data. The MLE of of the parameters of the model is defined to be
\begin{equation}
  \label{eq:MLE}
  \hat{\bm{\theta}}_{ML} = \argmax_{\bm{\theta} \in \bm{\Theta}}\mathcal{L}(\bm{\theta}; \mathcal{D})
\end{equation}
.

While the original MLE is defined in terms of the likelihood function
$\mathcal{L}(\bm{\theta}; \mathcal{D})$, it's often more practical to work with
the logarithm of this function, the log-likelihood function $\ell(\bm{\theta} ;
\mathcal{D})$. Using the common assumption of i.i.d datapoints, the joint
distribution becomes a product of individual probabilities for each datapoint,
\begin{equation}
  \label{eq:likelihood}
  \mathcal{L}(\bm{\theta} | \mathcal{D}) = p(\bm{x}_1, \dots, \bm{x}_n | \bm{\theta}) = \prod_i^n p(\bm{x}_i | \bm{\theta})
\end{equation}.
Using the log-likelihood we transform this product into a form involving sums
\begin{equation}
  \label{eq:log-likelihood}
  \ell(\bm{\theta} | \mathcal{D}) = \log \mathcal{L}(\bm{\theta} | \mathcal{D}) = \sum_i^n \log p(\bm{x}_i)
\end{equation}.
Besides from simplifying notation and calculation, it has the added benefit of
reducing the risk of arithmetic underflow due to the small magnitude of
individual probabilities\footnote{Also, with the use of the log-sum-exp-trick,
  \begin{equation*}
    \log \sum_{i=1}^n \exp(x_n) = \max_i x_i + \log \sum_{i=1}^n \exp(x_n - \max_i x_i)
  \end{equation*}
  this problem can be reduced even further}.

MLE estimators have a number of nice properties such as consistency and
asymptotic normality, however, the optimization problem itself is often
non-convex, making it hard to find the actual estimator\cite{CaseBerg:01}.

\subsection{Graphical models}
tool for probabilistic modelling is the notion of using
diagrams to specify the conditional relationships between random variables. A
graphical model is a diagrammatic way of specifying this relationship by
creating a Directed Acyclic Graph, a directed graph without any cycles. This
representation is called a \textit{Graphical Model} and provide a powerful way
to visualize the structure of the probabilistic model and also how to use and
abuse the structure of the model in order to infer variables in a computationally efficient way.

A graph in this setting consists of a set of \textit{vertices} connected by
\textit{edges}, following the notation and nomenclature of graph theory as used
in mathematics. While there are many different kinds of graphical models
depending on the type of graph structure used (directed graphical models,
undirected graphical models, factor graphs, etc.), we will only focus on the
subset of graphical models called directed graphical models.

Directed graphical models specify how the joint distribution of a set of random
variables $\mathcal{X}$ factors in a conditional manner. In general, the
relationship between a given directed graph and the corresponding distribution
over the variables in $\mathcal{X}$ is such that the join distribution defined
by the graph is given by the product, over all vertices of the graph, of a
conditional distribution for each vertex conditioned on the variables
corresponding to the parents of that vertex in the graph. So for a graph with
$K$ vertices, the joint distribution is give by
\begin{equation}
  \label{eq:dir_graph_model_dist}
  p(\mathcal{X}) = \prod_{k=1}^K p(x_k | \text{pa}(x_k))
\end{equation}
where $\text{pa}(x_k)$ is defined as the set of random variables corresponding
to the parent vertices of the random variable $x_k$.

Although a graphical model is completely defined in terms of it's vertex and
edge set, it is really the most powerful when visualized as a diagram. As an
example I will repeat the above formulation in the context of the directed
graphical model
\begin{figure}[H]
  \center
  \begin{tikzpicture}
    % Define nodes
    \node[latent] (a) {$a$} ;
    \node[latent, right=of a] (b) {$b$} ;
    \node[latent, below=of a, xshift=0.95cm] (c) {$c$} ;
    \node[obs, below=of c] (d) {$d$} ;

    % Connect the nodes
    \edge {a} {b, c} ;
    \edge {b} {c} ;
    \edge {c} {d} ;
  \end{tikzpicture}
\end{figure}
There are two different vertices in this graphical model, the greyed out vertex
indicates that the random variable is \textit{observed} such that it's value is
fixed and known. The white vertices indicates latent random variables which we
don't observe.

For this example we have the following factorisation of the joint distribution,
following the rules laid out in equation~\eqref{eq:dir_graph_model_dist},
\begin{equation*}
  p(a, b, c, d) = p(d | c)p(c | a, b)p(b | a)p(a)
\end{equation*}.

\subsection{Approximate Inference}

While MLE is in many ways the optimal way that we can fit the model, it's only
analytically and/or computationally feasible for very simple models which relies
simple transformations and tractable distributional relationships. In cases
where more powerful models are used it is very hard to find the MLE or even
local maximum to the likelihood $\mathcal{L}$, or equivalently the log-likelihood
$\ell$.

While in theory most of these problems can be resolved by MCMC
sampling, which also practically have been implemented in the way of
probabilistic programming with some
success\cite[Ch.~1]{brooks2011handbook}\cite{Carpenter_stan:a,
  journals/peerj-cs/SalvatierWF16}, most often this is too computationally
intensive and has a large cost in terms of time. Approximate inference forms an
alternative to MCMC for solving intractable densities by recasting this problem
into an optimization problem instead of a sampling one as in MCMC.

The most common setting is in latent variable models, where a latent
variable, often denoted by $\bm{z}$ is introduced to explain underlying causes
to the observed variables, such as different clusters for Mixture of
Gaussians or a lower-dimensional manifold in terms of the Factor Analysis
model\cite[page.~430-439, 583-586]{Bishop:2006}. Latent variable models which
rely on Gaussian distributions and linear relationships may be learned in an
exact manner in the context of the EM algorithm or its many variations which
guarantees parameters $\hat{\bm{\theta}}$ such that this point in the parameter
space will yield a local maximum of the likelihood function\cite{Dempster77maximumlikelihood, Neal98aview}.

As theory and computing power has progressed, so has the advances in more
powerful models. Compared to older models for which solutions often could be
found analytically, these models were too non-linear, non-gaussian and complex
to train in a straightforward manner, which can bee seen directly from the
numerous heuristics that exist with regards how to train deep neural
networks\cite{bengio_practical_2012, Larochelle:2009:EST:1577069.1577070}.

The problem setting of variational inference is a latent variable model such
that it can be split up into the observed variables $\bm{x}$ and latent
variables $\bm{z}$ such that
\begin{equation}
  \label{eq:latent_var_model}
  p(\bm{z}, \bm{x}) = p(\bm{z})p(\bm{x} | \bm{z})
\end{equation}. This also happens to cover the case of our generative model.

Technically, Approximate inference is a way of approximating
a complicated distribution $p(\bm{z} | \bm{x})$ by a distribution $q(\bm{z})$ belonging
to some constrained family of distributions $\mathcal{Q}$, for continuous
distributions often such that $\mathcal{Q} = \{\mathcal{N}(\bm{\mu},
\bm{\Sigma}) | \bm{mu} \in \mathbb{R}^d, \text{p.d \:} \bm{\Sigma} \in
\mathbb{R}^{d \times d}\}$. The goal is then to find the elementof $\mathcal{Q}$
that minimizes some distance from $p$ to $q$, most often the Kullback-Leibler
divergence,
\begin{equation}
  \label{eq:AI_optimal_element}
  q^*(\bm{z}) = \argmin_{q(\bm{z}) \in \mathcal{Q}} KL(q(\bm{z} || p(\bm{z} | \bm{x})))
\end{equation}.
This optimal $q^*$ may then be used as a pseudo-correct
distribution in order to calculate other statistics and quantities.

\section{Deep Learning}
Until recently the field of NLP were dominated by older machine learning
techniques utilising linear models trained over very high-dimensional and sparse
feature vectors. Recently the field has switched over to neural networks over
dense inputs instead using embeddings\cite[p.~1 - 2]{goldberg2015primer}.

What all neural networks have in common is that they are trying to find a
functional relationship for the data, with the specific form of the function
depending on the task. For NMT this reduces to finding the function $f : \bm{x}
\in \lang{X} \to \bm{y} \in \lang{Y}$ such that this $f$ maximizes the
likelihood $P(\bm{x} | \bm{y})$. Indeed many of the neural networks in existence
has been shown to be universal approximators, theoretically being able to
simulate a big set of nice functions\cite{Hornik:1989:MFN:70405.70408}.

\subsection{Multilayer Perceptron}
MLP's are neural networks represented by functional composition, where each
function is interpreted as a layer of the network. The original MLP can be defined in
terms of a recurrence relation such that if we have input vectors of the form $\bm{x} \in
\mathbb{R}^{d_{in}}$ and output vectors of the form $\bm{y} \in
\mathbb{R}^{d_{out}}$, then an MLP with $L$ layers have the functional form of
\begin{equation}
  f(\bm{x} | \bm{\theta}) = \sigma_L(\bm{W}_L \bm{z}_{L-1} + \bm{b}_{L})
\end{equation}
where for any $l \in \{2, \dots, L-1\}$
\begin{equation}
    \bm{z}_l = \sigma_l(\bm{W}_l \bm{z}_{l-1} + \bm{b}_l)
\end{equation}
and with the base case
\begin{equation}
  \bm{z}_1 = \sigma_1(\bm{W}_1 \bm{x} + \bm{b}_1)
\end{equation}.

$\bm{W}_l$ and $\bm{b}_l$ may be of any dimension as long as it is dimensionally consistent
with the input and output of the layer and conform to the original input and output
dimensions. In this case we have that the parameters of the network are all of
the biases and weights for the layers, $\bm{\theta} = \{(\bm{W}_l, \bm{b}_l)_{l
  = 1}^L\}$.

\subsection{Recurrent Neural Networks}
A recurrent neural network acts upon sequences to produce an output. It is a
special kind of technique called parameter tying where parameters are shared
over many time-steps.

\subsection{Convolutional Neural Networks}

\section{Natural Language Processing}

Humans use natural language every day to convey concepts and abstractions to each
other in an efficient manner. Compared to formal languages found in
mathematics and programming, the natural languages we use are often
ambiguous systems filled with rules and exceptions\cite{Rosenfeld00twodecades, sep-computational-linguistics}.

Natural Language Processing is an old field that for a long
time developed in parallel with the field of machine learning and
computational statistics which deals with how to process information coming from
human languages and is split up into several subfields such as Machine
Translation, statistical parsing and sentiment analysis\cite{sep-computational-linguistics}.

\subsection{Language model}
We define a sentence to be a vector of words $\bm{w}_{1:L} = (\mathsf{w}_1, \dots,
\mathsf{w}_l)^{\top}$ such that each word is an atomic element $\mathsf{w}_i \in
V$, where $V$ is the dictionary of words in our language. Then the joint
distribution of a word with respect to the underlying probability measure can be
rolled out using the probabilistic chain rule which is just repeated application
of the original product rule in equation \eqref{eq:product_rule}
\begin{equation}
  \label{eq:conditional_language_probability}
  P(\bm{w}_{1:L}) = \prod_{l = 1}^LP(\mathsf{w}_l | \mathsf{w}_1, \dots, \mathsf{w}_{l-1})
\end{equation}
where $\mathsf{w}_l$ is the $l$'th word of the sentence $\bm{w}_{1:L}$\cite{Bengio:2003:NPL:944919.944966}.

\subsection{Word embeddings}
Breaking down sentences at a word level and processing them into a form that
encodes information efficiently is a problem which has gained notorious
recognition, leading to algorithms such as word2vec and
Glove\cite{DBLP:journals/corr/abs-1301-3781, Pennington14glove:global,
  Mikolov:2013:DRW:2999792.2999959}. However, these techniques work less well in
a neural network setting where instead finding the best embedding jointly with
the parameters of the model is peqreferred \cite[p.~5-7]{goldberg2015primer}.

A straightforward way to represent the various words of the dictionary is as
one-hot-encoded vectors such that a word $\mathsf{w} \in V$, such that size of
$V$ is $|V|$, with an index $i$ given by its place in the dictionary sorted
alphabetically in descending order will have the vector representation
\begin{equation}
  \label{eq:one_hot_encoding}
  \bm{w} =
  \begin{bmatrix}
    0 \\
    \vdots \\
    0 \\
    1 \\
    0 \\
    \vdots \\
    0
  \end{bmatrix}
\end{equation}\cite[p.~6]{goldberg2015primer}
such that $\bm{w}_{j} = \delta_{ij}$.

While this is a form conceptually easy to understand, it fails to account for the
curse of dimensionality as the size of the vocabulary might grow to millions of
entries and the fact that the cosine similarity of two words $\mathsf{v}_1,
\mathsf{v}_2 \in V$ is zero unless they are the same word
\begin{equation}
  \label{eq:cosine_similarity}
  \cos_{similarity}(\bm{v}_1, \bm{v}_2) = \delta_{\mathsf{v}_1 \mathsf{v}_2}
\end{equation}. This means that no meaning is embedded in the vector space
except for the location in the sorted dictionary. Instead we would like to
associate each word in the vocabulary with a distributed \textit{word feature
  vector}, a dense, real-valued vector in $\mathbb{R}^m$. Express the joint probability
function \eqref{eq:conditional_language_probability} of word sequences of a
sentence in terms of the feature vectors of these words in the sequence and
simultaneously learn the word feature vectors and the parameters of the model
which dictates the form of the probability function, $\bm{\theta}$. After this
is done words which share similarities in some sense such as \texttt{Dog, Puppy}
would have a higher similarity score than unrelated concepts such as \texttt{Dog, Bulwark}\cite{Bengio:2003:NPL:944919.944966}.

We may represent this in a mathematical form by trying to find a linear map $C$
from any element $\mathsf{w} \in V$ such that $C(\mathbf{w}) \in \mathbb{R}^m$.
Using the usual canonical basis of an Euclidean space, we can express this
linear map in terms of a matrix $\bm{C} \in \mathbb{R}^{m \times |V|}$, thus the
word feature vector of the learned embedding can be represented by the matrix
multiplication $\bm{C}\bm{w}$.

\subsection{Neural machine translation}
For a long time the dominant paradigm within machine translation was to use
phrase based machine translation systems\cite{Koehn:2003:SPT:1073445.1073462,
  Koehn:2007:MOS:1557769.1557821}, however since a couple of year back,
modelling the word or character level directly with neural networks, so called
NMT has become the best performing method\cite{wolk_neural-based_2015, wu_googles_2016}.

Most NMT models work in terms of an encoder-decoder architecture where the
encoder extracts a fixed length representation $\bm{c}$, often called a context
vector, from a variable length input sentence $\bm{x} \in \lang{X}$, and the
decoder uses this representation to generate a correct translation $\bm{y} \in
\lang{Y}$ from this representation\cite{cho_properties_2014}.

\begin{figure}[H]
  \includestandalone[width=\textwidth]{./scripts/tikz_code/encoder_decoder}% 
%  \includegraphics[width=\textwidth]{encoderdecoder.pdf}
    \caption{Encoder decoder schematic}
  \label{fig:encoder_decoder}
\end{figure}

\section{Optimization}

\subsection{Problem formulation}

Numerical optimization has been applied to many parts of science and thus many
different algorithms have been developed to tackle different problems. However,
most of the theoretical results that exist deal with convex optimization, where
the function we are trying to optimize is particularly well-behaved, leading to
theoretical guarantees on the solution converged to. The surface of the function
we are trying to optimize in machine learning in general and deep learning in
particular is not well-behaved, being highly non-linear and non-convex, meaning
most theoretical results that exist do not apply to this domain\cite{choromanska_loss_2014}.

Nevertheless, while theoretical results are lacking, there has been substantial
advances in various optimisation techniques fit to attack the highly non-linear,
non-convex and high-dimensional optimization problems of learning in deep
models, particularly from the Stochastic Gradient Descent. This has led to
numerous gradient descent-like algorithms used machine learning\cite{Ruder17}.

\subsection{Stochastic Gradient Descent}

For a normal probabilistic machine learning problem, we have data $\mathcal{D}$
that we try to model with a model $\mathcal{M}$ parametrised by parameters
$\bm{\theta} \in \Theta$. The optimization problem can in the most general case be recast
as an effort to find the parameters $\bm{\theta}_{ML}$ that maximizes the
log-likelihood function \eqref{eq:log-likelihood}.

As this can't be found analytically we have to resort to numerical schemes to
find good candidates which hopefully will be close to the true solution
$\bm{\theta}_{ML}$. The first candidate is called Gradient Descent (GD).
GD approximates the gradient as if it was linear and takes steps to maximize the
probability through a hill-climbing scheme. If we let $\mathcal{D} =
\{\bm{z}_i\}_{i = 1}^n$ such that $\bm{z_i}$ is any general datapoint, $Q$ an
objective function that we try to maximize, such that $Q : \Theta \times
\mathcal{Z} \to \mathbb{R}$, then the update using GD looks as follows
\begin{equation}
  \label{eq:GD_update}
  \bm{\theta}_{t + 1} = \bm{\theta}_t - \gamma_t \frac{1}{n} \sum_{i = 1}^n \nabla_{\bm{\theta}} Q(z_i, \bm{\theta}_t)
\end{equation}.
While $\gamma_t$ may vary with $t$, it is often fixed for practical reasons. As
we can see GD takes into account all of the datapoints in the set, in some sense
updating the parameters with respect to the true linear approximation of the gradient.

However, with the huge size of data existing today, it is often not feasible to
calculate the update for all datapoints in the dataset due to the time it would
take and memory it would take to store the gradients.

Stochastic Gradient Descent is a similar algorithm to GD that instead of
calculating the gradient with respect to the whole dataset calculates an
approximate gradient, hence the word \textit{stochastic}, as it picks a random
subset of the dataset to update $\bm{\theta}$ with regards to. If we let $I_t$
be a random subset of the indices of $V$ of size $m$ then SGD does the
following update
\begin{equation}
  \label{eq:SGD_update}
  \bm{\theta}_{t + 1} = \bm{\theta}_t - \gamma_t \frac{1}{m} \sum_{i \in I_t} \nabla_{\bm{\theta}} Q(z_i, \bm{\theta}_t)
\end{equation}\cite{series/lncs/Bottou12}\cite[p.~240]{Bishop:2006}.

The stochasticity has been shown to act as a regularizer and several analyses of
this in terms of a Bayesian framework has been done, explaining the stationary
behaviour of SGD with constant learning rate after convergence
\cite{mandt_stochastic_2017, mandt_variational_2016}.

\subsection{ADAM}
SGD has found widespread use within the machine learning community due to strong
experimental results and ease of use, especially in deep learning. Lately
though, a number of alternatives have sprung up that aims to improve the vanilla
SGD, such as RMSProp\cite{Tieleman2012} and AdaGrad\cite{Duchi:EECS-2010-24}.

Adam takes inspiration from RMSProp and AdaGrad. Technically, Adam keeps an
exponential running average of the first and second order statistics of the
gradient, using these to calculate an adaptive learning rate.

As laid out in the original paper, ADAM operates on a stochastic objective
function $f(\bm{\theta})$, which in our case would be the probability averaged
over our input batch. $f_t(\bm{\theta})$ denotes this function over different
timesteps, equally $g_t = \nabla_{\bm{\theta}} f_t(\bm{\theta})$. $m_t$ and
$v_t$ denotes the first and second order moments of the gradients, however,
these are biased and are corrected in the algorithm.

The whole algorithm is as follows.

\begin{algorithm}
  \caption{ADAM}\label{ADAM}
  \begin{algorithmic}[1]
    \Require $\alpha$: Stepsize
    \Require $\beta_1, \beta_2 \in [0, 1)$: Exponential decay rates of the moment estimates
    \Require $f(\bm{\theta})$: Stochastic objective function with parameters
    $\bm{\theta}$
    \Require $\bm{\theta}_0$: Initial parameter vector
    \State $m_0 \gets 0$ (Initialize $1^{\text{st}}$ moment vector)
    \State $v_0 \gets 0$ (Initialize $2^{\text{nd}}$ moment vector)
    \State $t \gets 0$ (Initialize timestep)
    \While{$\bm{\theta}_t$ not converged}
    \State $t \gets t + 1$
    \State $g_t \gets \nabla_{\bm{\theta}}f_t(\bm{\theta}_{t-1})$ (get gradients w.r.t stochastic objective at timestep $t$)
    \State $m_t \gets \beta_1 \cdot m_{t-1} + (1 - \beta_1) \cdot g_t$ (Update biased first moment estimate)
    \State $v_t \gets \beta_2 \cdot v_{t-1} + (1 - \beta_2) \cdot g^2_t$ (Update biased second raw moment estimate)
    \State $\hat{m}_1 \gets m_t / (1 - \beta^t_1)$ (Compute bias-corrected first moment estimate)
    \State $\hat{v}_1 \gets v_t / (1 - \beta^t_2)$ (Compute bias-corrected second raw moment estimate)
    \State $\bm{\theta}_t \gets \bm{\theta}_{t-1} - \alpha \cdot \hat{m}_i/(\sqrt{\hat{v}_t} + \epsilon)$ (Update parameters)
    \EndWhile
    \Return $\bm{\theta}_t$ (Resulting parameters)
  \end{algorithmic}
\end{algorithm}

Experimentally Adam has shown very good results on training various deep
learning models such as MLP's, CNN's and RNN's so we will use it to train our
models throughout this dissertation\cite{kingma_adam:_2014}.
\chapter{Methods and Theory}
\label{MethodsCh}

This chapter deals with the theory needed in order to use VAE in our setting. It
will also deal with extending the theory and necessary calculations.

\subsection{Variational Inference}
Consider the following general graphical model of a latent variable $\bm{z}$ and
an observed variable $\bm{x}$ parametrised by $\bm{\theta}$:
\begin{figure}[H]
  \label{tikz:latent_variable_model}
  \centering
  \begin{tikzpicture}
    % Define nodes
    \node[latent] (z) {$\bm{z}$} ;
    \node[obs, below=of z] (x) {$\bm{x}$} ;
    \node[const, right=of z, xshift=-0.5] (theta) {$\bm{\theta}$} ;

    % Connect the nodes
    \edge {z} {x} ;
    \edge {theta} {z, x} ;

    % Plate
    \plate {zx} {(z)(x)} {$N$} ;
  \end{tikzpicture}
  \caption{Latent variable model}
\end{figure}
implying the joint distribution
\begin{equation}
  \label{eq:latent_variable_model}
  p(\bm{x}, \bm{z}) = p(\bm{x} | \bm{z}) p(\bm{z})
\end{equation}.

In order to fit the parameters of the model using optimisation of
$\ell(\bm{\theta} ; \mathcal{D})$ we need to be able to to find the gradient and
value of $\ell(\bm{\theta} ; \bm{x})$ to use ADAM. We can rewrite
the likelihood for a datapoint $\bm{x}$ as
\begin{equation}
  \label{eq:likelihood_integrate_out_latent}
  \mathcal{L}(\bm{\theta} ; \bm{x}) = \int_{\mathcal{Z}} p(\bm{x}, \bm{z}) \dif \bm{z}
\end{equation}
by using the sum rule of \ref{eq:sum_rule}. For many models, this integral is
either not available in a closed form or requires exponential time to compute
\cite{blei_variational_2017}. While for some special cases it is possible to
find a local optimum of the likelihood by using the EM algorithm
\cite{Dempster77maximumlikelihood}, in general it is too restrictive since we
need to be able to find $p(\bm{z} | \bm{x})$ analytically.

Generally it is possible to use sampling based techniques such as MCMC
\cite{brooks2011handbook} to sample from $p(\bm{z} | \bm{x})$ but in practice it is
often slow and depending on the problem may be less than ideal. Especially for
problems where we use complex models and have large datasets it's not possible
in practice to use MCMC \cite{blei_variational_2017}.

Variational inference approaches the problem of optimizing the intractable
log-likelihood by positing a variational family of distributions,
$\mathcal{Q}$ and introducing a variational distribution $q \in \mathcal{Q}$. The
log-likelihood is then bounded from below using concavity with the
$\log( \cdot )$ function in order to apply Jensen's inequality and the fact that
$q$ is a distribution:
\begin{align*}
  \log p_{\bm{\theta}}(\bm{x}) & = \log \int_{\mathcal{Z}} p_{\bm{\theta}}(\bm{z}, \bm{x}) \dif \bm{z}\\
                               & = \log \int_{\mathcal{Z}}q_{\bm{\varphi}}(\bm{z})  \frac{p_{\bm{\theta}}(\bm{x}, \bm{z})}{q_{\bm{\varphi}}(\bm{z})} \dif \bm{z} \\
                               & = \log \E_{q_{\bm{\varphi}}(\bm{z})}\left[\frac{p_{\bm{\theta}}(\bm{x}, \bm{z})}{q_{\bm{\varphi}}(\bm{z})}\right] \\
                               & \geq \E_{q_{\bm{\varphi}}(\bm{z})}\left[\log \frac{p_{\bm{\theta}}(\bm{x}, \bm{z})}{q_{\bm{\varphi}}(\bm{z})}\right] \\
                               & = \E_{q_{\bm{\varphi}}(\bm{z})}\left[\log p_{\bm{\theta}}(\bm{x}, \bm{z}) - \log q_{\bm{\varphi}}(\bm{z})\right] \\
\end{align*}.

We call the lower bound the Evidence Lower Bound Objective (ELBO),
\begin{equation}
  \label{eq:ELBO}
  \text{ELBO}(q_{\bm{\varphi}}) =
  \E_{q_{\bm{\varphi}}(\bm{z})}\left[ \log p_{\bm{\theta}}(\bm{x}, \bm{z}) - \log
    q_{\bm{\varphi}}(\bm{z}) \right]
\end{equation}
and we get the following inequality
\begin{equation}
  \label{eq:ELBO_inequality}
  \log p_{\bm{\theta}}(\bm{x}) \geq \text{ELBO}(q_{\bm{\varphi}})
\end{equation}.

Rewriting the ELBO in terms of the log-likelihood
\begin{equation}
  \label{eq:ELBO_decomposed}
  \text{ELBO}(q_{\bm{\varphi}}) = \log p_{\bm{\theta}}(\bm{x}) - \KL{q_{\bm{\varphi}}(\bm{z})}{p_{\bm{\theta}}(\bm{z} | \bm{x})}
\end{equation}
we see how the choice of $q_{\bm{\varphi}}$ affects the tightness of the
lower bound through the closeness to the true posterior $p_{\bm{\theta}}(\bm{z}
| \bm{x})$ in terms of KL-divergence \cite{blei_variational_2017}.

\section{VAE and SGVB}

The Variational Autoencoder by Kingma et. al \cite{kingma_auto-encoding_2013}
introduces a recognition model $q_{\bm{\varphi}}(\bm{z} | \bm{x})$ that serves as
an approximation to the true posterior $p_{\bm{\theta}}( \bm{z} | \bm{x})$. We let $q_{\bm{\varphi}}(\bm{z} | \bm{x}) \sim
\mathcal{N}(\bm{z}| \bm{\mu_{\bm{\varphi}}}(\bm{x}),
\bm{\sigma}_{\bm{\varphi}}(\bm{x}))$ such that the pair of vectors
$(\bm{\mu}_{\bm{\varphi}}(\bm{x}), \bm{\sigma}_{\bm{\varphi}}(\bm{x}))$ is the
output of a neural network, and let this Gaussian have a diagonal covariance
structure with the vector $\bm{\sigma}_{\bm{\varphi}}(\bm{x})$ as the diagonal.

If we consider the ELBO again we can rewrite it in terms that only involve known
functions and quantities
\begin{equation}
  \label{eq:ELBO_VAE}
  \text{ELBO}(q_{\bm{\varphi}}(\bm{z} | \bm{x})) = \E_{q_{\bm{\varphi}}(\bm{z} | \bm{x})}\left[ \log p_{\bm{\theta}}(\bm{x} | \bm{z}) \right] - \KL{q_{\bm{\varphi}}(\bm{z} | \bm{x})}{p_{\bm{\theta}}(\bm{z})}
\end{equation}.
Since this depends on both the parameters of the generative model and the
recognition model, $(\bm{\theta}, \bm{\varphi})$ we make this explicit and write
\begin{equation}
  \label{eq:VAE_ELBO}
  \mathcal{L}^{\text{ELBO}}(\bm{\theta}, \bm{\varphi} ; \bm{x}) = \E_{q_{\bm{\varphi}}(\bm{z} | \bm{x})}\left[ \log p_{\bm{\theta}}(\bm{x} | \bm{z}) \right] - \KL{q_{\bm{\varphi}}(\bm{z} | \bm{x})}{p_{\bm{\theta}}(\bm{z})}
\end{equation}.

Note that since for most models we are not able to get an analytical form of the
expectation over $q_{\bm{\varphi}}(\bm{z} | \bm{x})$ due to the non-linear
relationship between $\bm{x}$ and $\bm{z}$. Instead we sample $\bm{z}$ to
estimate this expectation using normal Monte Carlo estimation.

While it would be possible to differentiate and optimize the lower bound
$\mathcal{L}^{\text{ELBO}}(\bm{\theta}, \bm{\varphi} ; \bm{x})$ jointly with
respect to both the variational and generative parameters $\bm{\varphi}$ and
$\bm{\theta}$ by using a straightforward Monte Carlo gradient estimator with respect
to $q_{\bm{\varphi}}(\bm{z} | \bm{x})$, this gradient exhibits very high
variance and is impractical compared to the reparametrisation trick which we will introduce below \cite{kingma_auto-encoding_2013}.

\subsection{Reparametrisation Trick}
Under certain mild regularity conditions, for a chosen approximate posterior
$q_{\bm{\varphi}}(\bm{z} | \bm{x})$ we can express the distribution of $\bm{z}
\sim q_{\bm{\varphi}}(\bm{z} | \bm{x})$ in terms of a simpler distribution, in
our case a standard unit normal $\bm{\epsilon}$ and a differentiable
reparametrisation trick which will be introduced below
\cite{kingma_auto-encoding_2013} transformation $g_{\bm{\varphi}}(\bm{\epsilon},
\bm{x})$ such that
\begin{equation}
  \label{eq:reparametrisation_trick}
  \bm{z} = g_{\bm{\varphi}}(\bm{\epsilon}, \bm{x}), \quad \text{such that} \quad \bm{\epsilon} \sim \mathcal{N}(\bm{0}, \bm{I})
\end{equation}.
In our case we use \eqref{eq:sample_x_diag_covariance} in order to write
$g_{\bm{\varphi}}(\bm{\epsilon}, \bm{x}) = \bm{\mu}_{\bm{\varphi}}(\bm{x}) +
\bm{\sigma}_{\bm{\varphi}}(\bm{x}) \odot \bm{\epsilon}$.

This means that we can form Monte Carlo estimates of the ELBO by sampling
$\bm{z}$ through sampling $\bm{\epsilon}$ and using the differentiable
transformation
\begin{equation}
  \label{eq:SGVB}
    \E_{q_{\bm{\varphi}}(\bm{z} | \bm{x})} \left[ \log p_{\bm{\theta}}(\bm{x},  \bm{z}) - \log q_{\bm{\varphi}}(\bm{z} | \bm{x}) \right] \simeq \frac{1}{L} \sum_{l=1}^L \log p_{\bm{\theta}}(\bm{x},  \bm{z}^{l}) - \log q_{\bm{\varphi}}(\bm{z}^{l} | \bm{x})
\end{equation}
where $\bm{z}^{l} = g_{\bm{\varphi}}(\bm{\epsilon}^{l}, \bm{x})$ and
$\bm{\epsilon}^{l} \sim \mathcal{N}(\bm{0}, \bm{I})$.

Equation \eqref{eq:SGVB} is the Stochastic Gradient Variational Bayes estimator
of the ELBO, which we call $\mathcal{L}^{VAE}(\bm{\theta}, \bm{\varphi};
\bm{x})$ and is an unbiased estimator of the ELBO since we sample $\bm{z}$ from
the recognition model.

\subsection{AEVB algorithm}
The AEVB algorithm optimises the SGVB equation \eqref{eq:SGVB} in order to in
drive up the ELBO and push the lower bound often the log-likelihood up. If the
recognition model and the prior are chosen to be from appropriate distributions
it is possible to get an analytical form of the KL-divergence. Choosing
$p_{\bm{\theta}}(\bm{z})$ and $q_{\bm{\varphi}}(\bm{z} | \bm{x})$ to be
Gaussianly distributed enables us to write the SGVB in an alternative form which
typically has less variance than the SGVB form in equation \eqref{eq:SGVB}:
\begin{equation}
  \label{eq:SGVB_analytical_KL}
  \mathcal{L}^{VAE}(\bm{\theta}, \bm{\varphi}) = -\KL{q_{\bm{\varphi}}(\bm{z} | \bm{x})}{p_{\bm{\theta}}(\bm{z})} + \frac{1}{L} \sum_{l=1}^L \log p_{\bm{\theta}}(\bm{x} | \bm{z}^{l})
\end{equation}
where $\bm{z}^{l} = g_{\bm{\varphi}}(\bm{\epsilon}^{l}, \bm{x})$ and
$\bm{\epsilon}^{l} \sim \mathcal{N}(\bm{0},
\bm{I})$ \cite{kingma_auto-encoding_2013}. We let $L = 1$ following the advice
of Kingma et. al.

\section{Models}

Here we lay out the graphical models we will use in order to model language
generation. All of our models are latent variable models with the prior on
$\bm{z}$ being unit normal, $p(\bm{z}) = \mathcal{N}(\bm{z}| \bm{0}, \bm{I})$.

\subsection{Reconstruction}

The generative model is the same as in the figure for latent variable models
% \ref{tikz:latent_variable_model}
, except that the distribution of $\bm{z}$ does not depend on $\bm{\theta}$:
\begin{figure}[H]
  \center
  \label{tikz:reconstruction_model}
  \begin{tikzpicture}

    % Define nodes
    \node[latent] (z) {$\bm{z}$} ;
    \node[obs, below=of z] (x) {$\bm{x}$} ;
    \node[const, right=of z] (thetax){$\bm{\theta}$} ;
    
    % Connect the nodes
    \edge {z} {x} ;
    \edge {thetax} {x} ;

    % Make plate
    \plate {zx} {(z)(x)} {$N$};
    
  \end{tikzpicture}
  \caption{Reconstruction language model}
\end{figure}
where $\bm{z}$ is the latent variable representing the latent sentence structure
and the joint factorizes as $p_{\bm{\theta}}(\bm{x}, \bm{z}) =
p_{\bm{\theta}}(\bm{x} | \bm{z})p(\bm{z})$.

The reconstruction model is the same model specified in
\cite{kingma_auto-encoding_2013}. Bowman et. al has successfully showed that it
can be applied to natural language, learning meaningful representations of
language in the latent dimension \cite{bowman_generating_2015} while improving
on previous models such as RNNLM \cite{conf/icassp/MikolovKBCK11}.

\subsection{Translation}

Translation adds and extra observed variable $\bm{y}$, a sentence belonging to
$\lang{Y}$. The probabilistic model looks like:

\begin{figure}[H]
  \center
  \begin{tikzpicture}

    % Define nodes
    \node[latent] (z) {$\bm{z}$} ;
    \node[obs, below=of z, xshift=-1.0cm] (x) {$\bm{x}$} ;
    \node[obs, below=of z, xshift=1.0cm] (y) {$\bm{y}$} ;
    \node[const, left=of z, xshift=-0.5cm] (thetax) {$\bm{\theta}_{\bm{x}}$} ;
    \node[const, right=of z, xshift=0.5cm] (thetay) {$\bm{\theta}_{\bm{y}}$} ;
    
    % Connect the nodes
    \edge {z} {x} ;
    \edge {z} {y} ;
    \edge {thetax} {x} ;
    \edge {thetay} {y} ;

    % Make plate
    \plate {zxy} {(z)(x)(y)} {$N$} ;
    
  \end{tikzpicture}
  \caption{Translation language model}
\end{figure}

The graphical model implies that the joint distribution factorises as
\begin{equation}
  \label{eq:joint_generative}
  p(\bm{\bm{z}}, \bm{x}, \bm{y}) = p(\bm{z})p(\bm{x} | \bm{z})p(\bm{y} | \bm{z})
\end{equation}
and we make the following distributional assumption
\begin{align}
  p(\bm{x} | \bm{z}) & = \mathcal{N}(\mu_{\bm{\theta}_{\bm{x}}}(\bm{z}), \sigma_{\bm{\theta}_{\bm{x}}}(\bm{z})) \\
  p(\bm{y} | \bm{z}) & = \mathcal{N}(\mu_{\bm{\theta}_{\bm{y}}}(\bm{z}), \sigma_{\bm{\theta}_{\bm{y}}}(\bm{z}))
\end{align}

The notation $\mathcal{N}(\mu_{\bm{\theta}}(\bm{z}), \sigma_{\bm{\theta}}(\bm{z}))$
means that the mean and variance diagonal of the normal are represented by a
functional relationship $f: \bm{z} \mapsto (\bm{\mu}, \bm{\sigma})$ where $f$ is a function
represented a a learnable neural network with parameters $\bm{\theta}$.

The recognition model parametrises the probability distribution
$q_{\bm{\varphi}}(\bm{z} | \bm{x}, \bm{y})$. We are free to choose this
parametrisation and distributional form however we like, but the closer this is to the actual conditional distribution over the
latent, $p(\bm{z} | \bm{x}, \bm{y})$ in the KL sense, the tighter the
inequality in \eqref{eq:ELBO_inequality} will be as shown by equation
\eqref{eq:ELBO_decomposed}. We let the distributions be Gaussians to have
analytical tractability, 
\begin{align*}
  q_{\bm{\varphi}}(\bm{z} | \bm{x}, \bm{y}) & = q_{\bm{\varphi}_{\bm{x}}}(\bm{z} | \bm{x})q_{\bm{\varphi}_{\bm{y}}}(\bm{z} | \bm{y}) \\
                                            & = \mathcal{N}(\mu_{\bm{\varphi}_{\bm{x}}}(\bm{x}), \sigma_{\bm{\varphi}_{\bm{x}}}(\bm{x}))\mathcal{N}(\mu_{\bm{\varphi}_{\bm{y}}}(\bm{y}), \sigma_{\bm{\varphi}_{\bm{y}}}(\bm{y}))
\end{align*}.

This is again a Gaussian distribution
\begin{equation*}
  q(\bm{z} | \bm{x}, \bm{y}) = \mathcal{N}(\bm{z} | \mu(\bm{x}, \bm{y}), \sigma(\bm{x}, \bm{y}))
\end{equation*} which reduces to the expressions
\eqref{eq:joint_indep_normal_mean} for the mean and
\eqref{eq:joint_indep_normal_covariance} for the covariance as shown in the
background section on the Gaussian distribution.
Pulling all of these statements together, we get that the SGVB of the
translation case reduces to the equation
\begin{equation}
  \begin{split}
  \label{eq:translation_SGVB}
  \mathcal{L}^{VAE}(\bm{\theta}, \bm{\varphi}; \bm{x}, \bm{y}) = (\frac{1}{L} \sum_{l=1}^L \log(\mathcal{N}(\bm{x} | \mu_{\bm{\theta}_y}(\bm{z}), \sigma_{\bm{\theta}_y}(\bm{z}))) + \log(\mathcal{N}(\bm{y} | \mu_{\bm{\theta}_y}(\bm{z}), \sigma_{\bm{\theta}_y}(\bm{z})))) + \\ \frac{1}{2} \sum_{j=1}^J(1 + \log((\sigma_j^{(i)})^2) - (\mu_j^{(i)})^2 - (\sigma_j^{(i)})^2))
  \end{split}
\end{equation}

where the $\mu, \sigma$ is from the normal $\mathcal{N}(\bm{z} | \mu(\bm{x},
\bm{y}), \sigma(\bm{x}, \bm{y}))$, and the second expression on the right is the
analytical form of $\KL{q_{\bm{\varphi}}(\bm{z} | \bm{x}, \bm{y})}{p(\bm{z})}$ \cite{kingma_auto-encoding_2013}.

\subsection{Training}
Variational Autoencoders in NLP have a tendency to collapse the KL term in the
SGVB objective to zero, effectively being reduced a RNNLM. As the KL can be seen
as a measure of how much information is encoded in $\bm{z}$ this event makes the
use of VAE's to model language-agnostic features in the latent space useless.
One of the objectives is to make sure this KL-collapse doesn't happen.

We use KL-annealing where we anneal the ELBO by a pseudo-objective
\begin{equation}
  \label{eq:KL_annealing_ELBO}
  ELBO_{\beta_t} = \E_{q_{\bm{\varphi}}(\bm{z} | \bm{x})}[\log p_{\bm{\theta}}(\bm{x} | \bm{z})] - \beta_t * \KL{q_{\bm{\varphi}}(\bm{z} | \bm{x})}{p_{\bm{\theta}}(\bm{z})}
\end{equation}
where $\beta_t$ is an annealing term going from $0$ to $1$ as we train. We do
this by choosing a warm-up $\beta$ and let $\beta_t = \min(1, \frac{t}{\beta})$,
letting this warm up be done linearly until we recover the original ELBO
objective, similar to normal simulated annealing \cite{Kirkpatrick1983, kingma_auto-encoding_2013}.


\chapter{Experiments}
\label{ExperimentsCh}

\section{Data}

\subsection{Dataset}

The dataset we have chosen to evaluate the model on is the Europarl dataset
between languages English and French. Europarl is a dataset of the proceedings
of the European Parliament, comprising in total of the 11 official languages of
the European Union.

The dataset was chosen as the number of sentences for English and French is
enough to be able to generalise (the uncompressed size of the full dataset is
619MB, 288MB for English, 311MB for French) to new sentences, and furthermore has
established baseline for NMT in the form of BLEU scores for all the different
language pairs in the full dataset, English-French in
particular.\cite{koehn2005epc}

\begin{table}[t]
  \begin{center}
    \begin{tabular}{ |p{0.8\textwidth}| } 
      \hline
      \textbf{English}: I declare resumed the session of the European Parliament adjourned on Friday 17 December 1999, and I would like once again to wish you a happy new year in the hope that you enjoyed a pleasant festive period.\\
      \textbf{French}: Je déclare reprise la session du Parlement européen qui avait été interrompue le vendredi 17 décembre dernier et je vous renouvelle tous mes vux en espérant que vous avez passé de bonnes vacances.\\
      \hline
    \end{tabular}
    \caption{A randomly sampled sentence from the Europarl corpus}
  \end{center}
\end{table}

\subsection{Preprocessing}

The raw data is unfit for use directly with the model. For one thing the raw
data is in the form of strings and in order to leverage the mathematics easily
we need to translate the raw form into a form which take place in a
high-dimensional space instead, here $\mathbb{R}^N$. Equally we remove aspects
of the data that will make it harder for the model to learn due to sparsity and
other statistical peculiarities of the data and NLP in general.

Preprocessing the data we make the following simplifications

\subsubsection{Character Level}

% \item Only lowercase characters and common punctuation marks are considered to
%   be part of our alphabet.

%   \textbf{English Alpabet}: \texttt{abcdefghijklmnopqrstuvwxyz,.`'?!}\\
%   \textbf{French Alphabet}: \texttt{abcdefghijklmnopqrstuvwxyz,.`'?!âêîôûàèùéëïüç}\\

\subsubsection{Word Level}

The problem with words is that there exist an immense quantity of them, if even
just due to grammatical constructs (example: run, running, ran etc.). Similarly,
for any given point in time, words go in and out of use and this necessitates
choosing which words to include in the dictionary. The dictionary consists of
all of the words that we consider part of the language, everything not in the
dictionary are either too rare or for some other reason excluded from use.

\begin{itemize}
\item We only include sentences of length between 2 and 30. This makes sure that
  the model have long enough sentences such that it may learn from the
  dependencies between words, but short enough so that the parameters are able
  to capture the long-term dependencies of sentences.
\item We calculate the word frequencies in order to sort all of the words in the
  dataset in terms of how often it appear in absolute terms. This is then used
  to only retain the 80000 most common words. Words which are not part of this
  list gets replaced by an \texttt{<UNK>} token, specifying that it's an unknown
  word outside of the dictionary. This makes sure that only words which are
  prevalent enough such that the model can derive its relation to other words
  are part of the dictionary.
\item Newline characters were removed and replaced by \texttt{<EOS>},
  end-of-sentence tokens, signifying the end of a sentence.
\item In parts where we have a dataset of 2 or more language sentences in
  parallel, we make sure that both of the the languages both satisfy the above criteria.
\end{itemize}

It is important to note that due to how languages differ, even though the
dataset might consist of sentence pairs this will still not mean that in general
both of the languages will have the same dictionary of words. Partially this is
due to the different ways that languages are built up when expressing meaning,
but on a more basic level, there are no bijection between languages as words
have slightly different meaning and contexts, with some words only existing in
one language but not the other.

\section{Scores}

We will evaluate our models on a variety of scores:

\begin{description}
\item[ELBO] ELBO is the lower bound of the actual log-likelihood of the observed
  data
  \begin{equation*}
    \sum_{i=1}^{N} \log p_{\theta}(\bm{x}_i)
  \end{equation*}
\item[Qualitative] Since natural language is not a formal in the sense that it
  is ambiguous, inconsistent and with exceptions to rules; any of these scores
  will be imperfect insofar as taking into account the feel of the generated
  sentences. Due to this we will inspect the sentences manually.
\item[BLEU] BLEU compares the generated sentences with sentences translated by
  professional translators, yielding a score telling us how well the generated
  translation does in relation to the translated benchmarks for each sentence.
\item[KL] Part of our investigation is about building models that take into
  account the latent space, enforcing the model to encode the information in the
  latent variable z instead of the encoder/decoder part. Luckily, we have a
  quantitative measure of this, the KL divergence between the prior and the
  posterior q-distribution over $\bm{z}$,
  \begin{equation*}
    KL[q_{\phi}(\bm{z} | \bm{x}) || p_{\theta}(\bm{z})]
  \end{equation*}
  , where the KL is a measure of how much information is put into the
  q-distribution compared to just using the prior isotropic gaussian over
  $\bm{z}$, $p_{\theta}(\bm{z})$.
\end{description}

\section{layout}

We will perform the following experiments, building up the order of doing them
from least complex to more complex.

We first have different models, depending on how we choose

\begin{description}
\item[Recognition model]
  \begin{itemize}
  \item WaveNet
  \item RNN
  \item MLP
  \end{itemize}
\item[Factorisation of Rec model]
  \begin{itemize}
  \item Independence of $\bm{x}, \bm{y}$
  \item diagonal sigma
  \item opposite of these
  \end{itemize}
\item[Generative model]
  \begin{itemize}
  \item AUTR
  \item WaveNet
  \item RNN
  \end{itemize}
\end{description}.

We then use these models on different types of language modelling:

\begin{description}
\item[Reconstruction]
  \begin{figure}[h]
    \center
    \begin{tikzpicture}

      % Define nodes
      \node[latent] (z) {$z$} ;
      
      \node[obs, below=of z] (x) {$x$} ;
      
      \edge {z} {x} ;
    \end{tikzpicture}
  \end{figure}
\item[Translation]
  \begin{figure}[h]
    \center
    \begin{tikzpicture}

      % Define nodes
      \node[latent] (z) {$z$} ;
      
      \node[obs, below=of z, xshift=-1cm] (x) {$x$} ;
      \node[obs, below=of z, xshift=1cm] (y) {$y$} ;
      
      \edge {z} {x, y} ;
    \end{tikzpicture}
  \end{figure}
\end{description}
While using the recognition model to do translation (We let $\bm{x}$ encode all
information about $\bm{z}$, and then see what the generated $\bm{x}, \bm{y}$
correspond to).

\chapter{Conclusions}
\label{ConclusionsCh}

What have we learned from all of this?

\addcontentsline{toc}{chapter}{Appendices}

% The \appendix command resets the chapter counter, and changes the chapter numbering scheme to capital letters.
%\chapter{Appendices}
\appendix
\chapter{An Appendix About Stuff}
\label{appendixlabel1}
(stuff)

\chapter{Another Appendix About Things}
\label{appendixlabel2}
(things)

\chapter{Colophon}
\label{appendixlabel3}
\textit{This is a description of the tools you used to make your thesis. It helps people make future documents, reminds you, and looks good.}

\textit{(example)} This document was set in the Times Roman typeface using \LaTeX\ and Bib\TeX , composed with a text editor. 
 % description of document, e.g. type faces, TeX used, TeXmaker, packages and things used for figures. Like a computational details section.
% e.g. http://tex.stackexchange.com/questions/63468/what-is-best-way-to-mention-that-a-document-has-been-typeset-with-tex#63503

% Side note:
%http://tex.stackexchange.com/questions/1319/showcase-of-beautiful-typography-done-in-tex-friends 
 
% You could separate these out into different files if you have
% particularly large appendices.

% This line manually adds the Bibliography to the table of contents.
% The fact that \include is the last thing before this ensures that it
% is on a clear page, and adding it like this means that it doesn't
% get a chapter or appendix number.
\addcontentsline{toc}{chapter}{Bibliography}

% Actually generates your bibliography.
\bibliography{Dissertation}
\bibliographystyle{abbrv}

% All done. \o/
\end{document}
