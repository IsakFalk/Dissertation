% I may change the way this is done in a future version, 
%  but given that some people needed it, if you need a different degree title 
%  (e.g. Master of Science, Master in Science, Master of Arts, etc)
%  uncomment the following 3 lines and set as appropriate (this *has* to be before \maketitle)
\makeatletter
\renewcommand {\@degree@string} {Master of Science}
\makeatother

\title{Latent Variable Models in Neural Machine Translation}
\author{John Isak Texas Falk}
\department{The Centre for Computational Statistics and Machine Learning}

\maketitle
\makedeclaration

\begin{abstract} % 300 word limit
Neural Machine translation is a direction in automated translation which learns
the mapping from the source to the target language directly in an end-to-end
fashion using the framework of neural networks. As NMT hinges on advances in
neural network methods and architectures, it is able to directly use
improvements there in its own domain.

We use recent improvements in latent variables models in order to train a
completely probabilistic generative model with a latent variable representing a
language-agnostic representation of a sentence mapping through deep neural
networks to two different output languages. Following recent advances in
training deep generative latent variable models we approximate the posterior of
the latents given output with a recognition model, mimicking a VAE. Following
the SGVB method to find a stochastic lower bound to the true log-likelihood of
the observed data, we train the parameters of the generative model and the
variational recognition model jointly to optimise this bound.

Since the recognition model acts as a pseudo-posterior (it approximates it given
constraints on the distributional form of the recognition model) we can use this
to translate from one language to another by finding the posterior
q-distribution over z and then from sampled z find the most likely output of the
languages.
\end{abstract}

\begin{acknowledgements}
I would like to thank the help from my supervisor Harshil Shah for helping me
make this thesis possible, and my parents, for always being there for me.
\end{acknowledgements}

\setcounter{tocdepth}{2} 
% Setting this higher means you get contents entries for
%  more minor section headers.

\tableofcontents
\listoffigures
\listoftables