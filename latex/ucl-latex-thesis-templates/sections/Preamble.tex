% I may change the way this is done in a future version, 
%  but given that some people needed it, if you need a different degree title 
%  (e.g. Master of Science, Master in Science, Master of Arts, etc)
%  uncomment the following 3 lines and set as appropriate (this *has* to be before \maketitle)
\makeatletter
\renewcommand {\@degree@string} {Master of Science}
\makeatother

\title{Neural Multi-Language Generation using Variational Autoencoders}
\author{John Isak Texas Falk}
\department{The Centre for Computational Statistics and Machine Learning}

\maketitle
%\makedeclaration

\leavevmode\thispagestyle{empty}\newpage

\begin{abstract}
  A language model is a statistical model that parametrise a distribution over
  sentences. Latent variable models have found widespread use throughout machine learning
  due to its ability to encode information about data in an underlying smaller space. Applying latent
  variable models to language models enable us to generate novel sentences
  from a continuous stochastic latent variable. While powerful, latent variable
  models are notoriously hard to train due to intractability of the
  log-likelihood. The VAE framework lets us to choose recognition model and a
  generative model in order to learn a deep latent variable models for text, avoiding having to optimise the log-likelihood directly. We
  introduce the VAE framework in an NLP setting and apply it to multi-language text generations using neural networks. We extend the theory to the case of
  two languages sharing a latent space. Implementing the theory and applying it to parallel language data we
  show how stronger generative models result in better sentences without
  collapsing the latent distribution to the prior, enabling us to generate coherent sentences in both languages and also perform language reconstruction using
  the recognition model. 
\end{abstract}

\setcounter{tocdepth}{2} 
% Setting this higher means you get contents entries for
%  more minor section headers.

\tableofcontents
\listoffigures
