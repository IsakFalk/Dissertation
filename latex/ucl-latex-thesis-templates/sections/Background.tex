\chapter{Background Knowledge}
\label{BackgroundKnowledgeCh}

From here on I will assume familiarity with some concepts which will be
important for the experiments that we will conduct and analyse.

\section{Probability Theory and Statistics}

\subsection{Probability Theory as a Reasoning System}
Although often seen as a self-contained mathematical theory of stochastic
systems and reasoning about the random, probability theory has a prominent role
within machine learning since it gives us a principled way of reasoning about
the world. As proved by Cox, if we are to accept that any theory of reasoning is
to satisfy the Cox desiderata,
\begin{enumerate}
\item Degrees of plausibility are represented by real numbers.
\item Qualitative correspondence with common sense
\item If a conclusion can be reasoned out in more than one way, then
  every possible way must lead to the same result.
\end{enumerate},
then we must accept that this theory is isomorphic to probability theory and
effectively the same.

This is a formal way of stating that probability theory is the best way to
organize our reasoning if we want to make sure that we are consistent in the way
we reason.\cite[p.~3-23]{jaynes2003probability}

\subsection{Rules and Theorems}
Rules of probability that will be useful to us are the following axioms
\begin{description}
\item[Unit volume]
  \begin{equation}
    \label{eq:unit_vol_prob_axiom}
    \int_{\mathcal{X}}p(X) = 1
  \end{equation}
\item[Non-negativity]
  \begin{equation}
    \label{eq:non_neg_of_prob}
    p(X) \geq 0
  \end{equation}
\end{description}
where $\mathcal{X}$ is the domain of $X$ and the generalized integral is interpreted as
the Lebesgue integral if $X$ is continuous and as a sum over the possible values
of $X$ if it is discrete.

I will assume the notion of a random variable $X$ in an intuitive sense.
However, this $X$ might be represented in many forms and in our case there is
not reason to not be able to put probabilities over sentences, words and/or
characters and conditioning thereof. Similarly I will assume familiarity with
the different notions of continuous, discrete and categorical random variables.
Finally 

Most manipulation of statements about random variables can be stated as a consequence
of the two following fundamental rules
\begin{description}
\item[Sum rule]
  \begin{equation}
    \label{eq:sum_rule}
    p(X) = \int_{\mathcal{Y}}p(X, Y)
  \end{equation}
\item[Product rule]
  \begin{equation}
    \label{eq:product_rule}
    p(X, Y) = p(Y | X)p(X)
  \end{equation}
\end{description}
such that $X, Y$ are two random variables defined on the domains $\mathcal{X},
\mathcal{Y}$. The integral $\int_{\mathcal{Y}}$ is to be understood in the
general sense, if $Y$ is continuous it is the ordinary Lebesgue integral as
commonly used throughout mathematics and calculus, while if $Y$ is discrete then
it is the sum over the possible values of $Y$. $p(X, Y)$ is the joint
distribution of $X$ and $Y$, $p(Y | X)$ is the probability of $Y$ conditioned on
n$X$ and $p(X)$ is the marginal distribution of $X$. Using these rules it is easy
to Bayes theorem, one of the most integral (and simple) theorem of probability
theory

\fbox{
  \begin{minipage}{0.8\textwidth}
    Bayes Theorem

    \begin{equation}
      \label{eq:Bayes_thm}
      p(Y | X) = \frac{p(X | Y)p(Y)}{p(X)}
    \end{equation}
  \end{minipage}
  \hfill
}

Two very important operations involving probabilities of random variables are
those of \textit{Expectation} and \textit{Covariance}. These take as input a
function $f$ and maps to the real number line $\mathbb{R}$, and are defined
implicitly with regards to some random variable $X$ and its 
probability distribution $p(X)$.
\begin{description}
\item[Expectation]
  \begin{equation}
    \label{eq:expectation}
    \E_X[f] = \int_{\mathcal{X}} p(x)f(x)
  \end{equation}
\item[Covariance]
  \begin{equation}
    \label{eq:covariance}
    \Cov(X, Y) = \E_{XY}[(X - \E_X[X])(Y - \E_Y[Y])]
  \end{equation}
\end{description}
We then define the variance operator as
\begin{equation}
  \Var(X) = \Cov(X, X)
\end{equation}

The generalisation from $f: \mathcal{X} \to \mathbb{R}$ to $f: \mathcal{X} \to
\mathbb{R}^n$ is defined in the straightforward manner such that if $\bm{f} =
f(X)$ then
\begin{equation*}
  \E_X
  \begin{bmatrix}
    \bm{f}_1 \\
    \vdots \\
    \bm{f}_n \\
  \end{bmatrix} =
  \begin{bmatrix}
    \E_X \bm{f}_1 \\
    \vdots \\
    \E_X \bm{f}_n \\
  \end{bmatrix}
\end{equation*}
similarly $\Cov(\bm{f})$ is a $D \times D$-dimensional matrix where
$\Cov(\bm{f})_{i,j} = \Cov(\bm{f}_i, \bm{f}_j)$.

\subsection{The Gaussian Distribution}
A very common distribution in machine learning and statistics in general is the
Gaussian distribution, also called the Normal distribution. The Gaussian
distribution satisfies some properties that makes it an ideal candidate from a
modeling perspective including the Central Limit Theorem which says that sums of
independent random variables of finite mean and variance will tend to a normal
distribution, and the fact that the joint distribution of many random variables
that are Gaussianly distributed is itself Gaussian which means that
conditionals, posteriors and other distributions are themselves Gaussian when we
deal with Gaussian random variables.

For a $D$-dimensional vector $\bm{X}$, the multivariate Gaussian distribution
takes the form
\begin{equation}
  \label{eq:Gaussian_dist}
  \mathcal{N}(\bm{x} | \bm{\mu}, \bm{\Sigma}) = \frac{1}{(2\pi)^{D/2}}\frac{1}{|\bm{\Sigma}|^{1/2}}\exp\left( -\frac{1}{2}(\bm{x} - \bm{\mu})^T\bm{\Sigma}^{-1}(\bm{x} - \bm{\mu})\right)
\end{equation}
where $\bm{\mu}$ is a $D$-dimensional mean vector, $\bm{\Sigma}$ is a $D \times
D$ dimensional covariance matrix, and $|\bm{\Sigma}|$ denotes the determinant of
$\bm{\Sigma}$. The meaning of these parameters can be shown to correspond to the
operations of the expected value and covariance of $\bm{x}$,
$\E_{\bm{x}}[\bm{x}] = \bm{\mu}$ and $\Cov(\bm{x}) = \bm{\Sigma}$.

The Gaussian distribution can be seen as a unit $D$-dimensional cube which is
translated, sheared and rotated, giving rise to the fact that we can write any
Gaussianly distributed random variable $\bm{x} \sim \mathcal{N}(\bm{x} |
\bm{\mu}, \bm{\Sigma})$ as a linear combination of a unit Gaussian random
variable $\bm{z} \sim \mathcal{N}(\bm{x} | \bm{0}_D, \bm{I}_{D \times D})$. If
we let $\bm{\Lambda} \bm{\Lambda} = \bm{\Sigma}$ be the Cholesky
decomposition\cite[p.~100-102]{Press:2007:NRE:1403886} of $\bm{\Sigma}$, then we
also have that
\begin{equation*}
  p(\bm{x}) = p(\bm{\mu} + \bm{\Lambda}\bm{z})
\end{equation*}
. If we further assume that $\bm{x}$ is parametrised by $\bm{\mu}$ and
$\bm{\Sigma}$ such that $\bm{\Sigma}$ is diagonal positive definite with
diagonal $\bm{\sigma}$, then $\bm{\Sigma} = \bm{\sigma} \odot \bm{I}_{D \times
  D}$. Finally this means that if we want to sample a random variable $\bm{x}$
with diagonal covariance structure, then we can do this by sampling a unit
normal $\bm{z}$ which we then transform, which we can express as
\begin{equation}
  \label{eq:sample_x}
  \bm{x} = \bm{\mu} + \bm{\sigma} \odot \bm{z} \sim \mathcal{N}(\bm{\mu}, \bm{\sigma} \odot \bm{I}_{D \times D})
\end{equation}

As the Gaussian distribution is part of the exponential family, the density of
joint distribution of iid Gaussian variables are themselves Gaussian distributed
where the natural parameters of this joint distribution is the sum of the
natural parameters of each random variable in the joint. In particular for the
Gaussian distribution, this means that if we have a collection of iid gaussian
random variables $\{\bm{x}_i)\}_i^n$, such that $\bm{x}_i \sim
\mathcal{N}(\bm{x}_i | \bm{\mu}_i, \bm{\Sigma}_{i})$, then the joint
can be found to be Gaussian distributed as
\begin{equation}
  \label{eq:join_dist_Gaussian}
  \mathcal{N}(\bm{\mu}, \bm{\Sigma})
\end{equation},
where
\begin{align*}
  \bm{\Sigma} & = \left( \sum_i^n \bm{\Sigma}_i^{-1} \right)^{-1}\\ 
  \bm{\mu} & = \bm{\Sigma}\left( \sum_i^n \bm{\Sigma}^{-1} \bm{\mu}_i \right)
\end{align*}.\cite[p.~78-84]{Bishop:2006}

In the case of two random variables distributed according to the form as laid
out in \ref{eq:sample_x}, $\bm{x} \sim
\mathcal{N}(\mathcal{N}(\bm{\mu}_{\bm{x}}, \bm{\sigma}_{\bm{x}} \odot \bm{I}))$
and $\bm{y} \sim
\mathcal{N}(\mathcal{N}(\bm{\mu}_{\bm{y}}, \bm{\sigma}_{\bm{y}} \odot \bm{I}))$
we have that the resulting distribution $p(\bm{x}, \bm{y}) = p(\bm{x})p(\bm{y})$ is distributed such that
\begin{equation}
  \label{eq:twin_joint_diag_cov}
  p(\bm{x}, \bm{y}) = \mathcal{N}(\bm{\mu}_{\bm{x}, \bm{y}}, \bm{\sigma}_{\bm{x}, \bm{y}})
\end{equation}
where
\begin{align*}
  \bm{\sigma}_{\bm{x}, \bm{y}} & = \frac{1}{\bm{\sigma}_{\bm{x}}^{-1} + \bm{\sigma}_{\bm{y}}^{-1}} \\
  \bm{\mu}_{\bm{x}, \bm{y}} & = \frac{\bm{\sigma}_{\bm{x}}^{-1}\bm{\mu}_{\bm{x}} + \bm{\sigma}_{\bm{y}}^{-1}\bm{\mu}_{\bm{y}}}{\bm{\sigma}_{\bm{x}}^{-1} + \bm{\sigma}_{\bm{y}}^{-1}} \\
\end{align*}.\ref{appendixproofs}

\subsection{Maximum Likelihood Estimation}
An often used estimator for parameters is the \textit{Maximum Likelihood
  Estimator}, hereafter called the MLE.

Assume we have a model $\mathcal{M}$ parametrised by $\bm{\theta}$ constrained
to live in the parameter space $\bm{\Theta}$. Given data $\mathcal{D}$ we want
to be able to fit the parameters $\bm{\theta}$ such that these somehow explains
the data, and thus should be able to work on new data in a manner that
generalises well. The MLE of of the parameters of the model is defined to be
\begin{equation*}
  \hat{\bm{\theta}}_{ML} = \argmax_{\bm{\theta} \in \bm{\Theta}}\mathcal{L}(\bm{\theta}; \mathcal{D})
\end{equation*}
.

While the original MLE is defined in terms of the likelihood function
$\mathcal{L}(\bm{\theta}; \mathcal{D})$, it's often more practical to work with
the logarithm of this function, the log-likelihood function $\ell(\bm{\theta} ;
\mathcal{D})$. Using the common assumption of i.i.d datapoints, the joint
distribution becomes a product of individual probabilities for each datapoint,
\begin{equation}
  \label{eq:likelihood}
  \mathcal{L}(\bm{\theta} | \mathcal{D}) = p(\bm{x}_1, \dots, \bm{x}_n | \bm{\theta}) = \prod_i^n p(\bm{x}_i | \bm{\theta})
\end{equation}.
Using the log-likelihood we transform this product into a form involving sums
\begin{equation}
  \label{eq:log-likelihood}
  \ell(\bm{\theta} | \mathcal{D}) = \log \mathcal{L}(\bm{\theta} | \mathcal{D}) = \sum_i^n \log p(\bm{x}_i)
\end{equation}.
Besides from simplifying notation and calculation, it has the added benefit of
reducing the risk of arithmetic underflow due to the small magnitude of
individual probabilities\footnote{Also, with the use of the log-sum-exp-trick,
  \begin{equation*}
    \log \sum_{i=1}^n \exp(x_n) = \max_i x_i + \log \sum_{i=1}^n \exp(x_n - \max_i x_i)
  \end{equation*}
  this problem can be reduced even further}

In a sense, the MLE is the best fit to the data given that we put a flat prior
over the parameters $\bm{\theta}$, that is we let the data speak for itself. The
MLE only takes into account the information given by the available data
$\mathcal{D}$ and thus makes a choice of $\bm{\theta}$ based upon the
information in the data set and nothing else. Besides this intuitive
justification of the estimator, from a frequentist perspective it comes with a
number of advantageous properties that we like an estimator to have, making it a
natural candidate for training models.

The main problem with MLE's are that they are defined implicitly in terms of the
likelihood function of the data. In that sense every setting is different as
there is no straightforward way to solve for the MLE as we in most cases have to
resort to numerical solutions to try to find the global maximum of
$\mathcal{L}(\bm{\theta}; \mathcal{D})$, a function which is often non-linear,
multimodal and with a complex surface, leading to local maximums being found
instead of the global maximum in many cases.\cite{CaseBerg:01}

\subsection{Graphical models}
A very convenient tool for probabilistic modelling is the notion of using
diagrams to specify the conditional relationships between random variables. A
graphical model is a diagrammatic way of specifying this relationship by
creating a Directed Acyclic Graph, a directed graph without any cycles. This
representation is called a \textit{Graphical Model} and provide a powerful way
to visualize the structure of the probabilistic model and also how to use and
abuse the structure of the model in order to infer variables in a
computationally efficient way.

A graph in this setting consists of a set of \textit{vertices} connected by
\textit{edges}, following the notation and nomenclature of graph theory as used
in mathematics. While there are many different kinds of graphical models
depending on the type of graph structure used (directed graphical models,
undirected graphical models, factor graphs, etc.), we will only focus on the
subset of graphical models called directed graphical models.

Directed graphical models specify how the joint distribution of a set of random
variables $\mathcal{X}$ factors in a conditional manner. In general, the
relationship between a given directed graph and the corresponding distribution
over the variables in $\mathcal{X}$ is such that the join distribution defined
by the graph is given by the product, over all vertices of the graph, of a
conditional distribution for each vertex conditioned on the variables
corresponding to the parents of that vertex in the graph. So for a graph with
$K$ vertices, the joint distribution is give by
\begin{equation}
  \label{eq:dir_graph_model_dist}
  p(\mathcal{X}) = \prod_{k=1}^K p(x_k | \text{pa}(x_k))
\end{equation}
where $\text{pa}(x_k)$ is defined as the set of random variables corresponding
to the parent vertices of the random variable $x_k$.

Although a graphical model is completely defined in terms of it's vertex and
edge set, it is really the most powerful when visualized as a diagram. As an
example I will repeat the above formulation in the context of the directed
graphical model
\begin{figure}[H]
  \center
  \begin{tikzpicture}
    % Define nodes
    \node[latent] (a) {$a$} ;
    \node[latent, right=of a] (b) {$b$} ;
    \node[latent, below=of a, xshift=0.95cm] (c) {$c$} ;
    \node[obs, below=of c] (d) {$d$} ;

    % Connect the nodes
    \edge {a} {b, c} ;
    \edge {b} {c} ;
    \edge {c} {d} ;
  \end{tikzpicture}
\end{figure}
There are two different vertices in this graphical model, the greyed out vertex
indicates that the random variable is \textit{observed} such that it's value is
fixed and known. The white vertices indicates latent random variables which we
don't observe.

For this example we have the following factorisation of the joint distribution,
following the rules laid out in equation~\ref{eq:dir_graph_model_dist},
\begin{equation*}
  p(a, b, c, d) = p(d | c)p(c | a, b)p(b | a)p(a)
\end{equation*}.

\subsection{Approximate Inference}

While MLE is in many ways the optimal way that we can fit the model, it's only
analytically and/or computationally feasible for very simple models which relies
simple transformations and tractable distributional relationships. In cases
where more powerful models are used it is very hard to find the MLE or even
local maximum to the likelihood $\mathcal{L}$, or equivalently the log-likelihood
$\ell$.

While in theory most of these problems can be resolved by MCMC
sampling, which also practically have been implemented in the way of
probabilistic programming with some
success\cite[Ch.~1]{brooks2011handbook}\cite{Carpenter_stan:a,
  journals/peerj-cs/SalvatierWF16}, most often this is too computationally
intensive and has a large cost in terms of time. Approximate inference forms an
alternative to MCMC for solving intractable densities by recasting this problem
into an optimization problem instead of a sampling one as in MCMC.

Approximate inference form a group of methods for approximating probability
densities and is widely used to approximate posterior densities for Bayesian
models. Compared to MCMC it tends to be faster and easier to scale to larger
datasets, but lacks in generality as every new problem has to be solved again
within the approximate inference framework.

The most common setting is in latent variable models, where a latent
variable, often denoted by $\bm{z}$ is introduced to explain underlying causes
to the observed variables, such as different clusters for Mixture of
Gaussians or a lower-dimensional manifold in
terms of the Factor Analysis model\cite[page.~430-439, 583-586]{Bishop:2006}.
Latent variable models which rely on Gaussian distributions and linear
relationships may be learned in an exact manner in the context of the EM
algorithm or its many variations which guarantees parameters $\hat{\bm{\theta}}$
such that this point in the parameter space will yield a local maximum of the
likelihood function\cite{Dempster77maximumlikelihood, Neal98aview}.

As theory and computing power has progressed, so has the advances in more
powerful models. Compared to older models for which solutions often could be
found analytically, these models were too non-linear, non-gaussian and complex
to train in a straightforward manner, which can bee seen directly from the
numerous heuristics that exist with regards how to train deep neural
networks\cite{bengio_practical_2012, Larochelle:2009:EST:1577069.1577070}. As
neural network architectures are becoming commonplace in bayesian modelling,
techniques to train these models has also become more prevalent.

The problem setting of variational inference is a latent variable model such
that it can be split up into the observed variables $\bm{x}$ and latent
variables $\bm{z}$ such that
\begin{equation}
  \label{eq:latent_var_model}
  p(\bm{z}, \bm{x}) = p(\bm{z})p(\bm{x} | \bm{z})
\end{equation}.
Indeed this describes a plethora of different models showing how broadly the
variational framework can be applied. Variational inference has enjoyed much
success in the past over other domains as well\cite{beal2003}, but we will focus
on this special case.

Technically, Approximate inference is a way of approximating
a complicated distribution $p(\bm{z} | \bm{x})$ by a distribution $q(\bm{z})$ belonging
to some constrained family of distributions $\mathcal{Q}$, for continuous
distributions often such that $\mathcal{Q} = \{\mathcal{N}(\bm{\mu},
\bm{\Sigma}) | \bm{mu} \in \mathbb{R}^d, \text{p.d \:} \bm{\Sigma} \in
\mathbb{R}^{d \times d}\}$. The goal is then to find the elementfof $\mathcal{Q}$
that minimizes some distance from $p$ to $q$, most often the Kullback-Leibler
divergence,
\begin{equation}
  \label{eq:AI_optimal_element}
  q^*(\bm{z}) = \argmin_{q(\bm{z}) \in \mathcal{Q}} KL(q(\bm{z} || p(\bm{z} | \bm{x})))
\end{equation}.
This optimal $q^*$ may then be used as a pseudo-correct
distribution in order to calculate other statistics and quantities.

This also means that we have recast the original MLE into a simpler optimisation
problem involving terms which lends itself to analytical solutions and
computational tractability. For MLE we are interested in optimizing the
log-likelihood
\begin{equation*}
  p(\bm{x} | \bm{\theta}) = \int_{\mathcal{Z}}p(\bm{x} | \bm{z})p(\bm{z})
\end{equation*}. For many models this is not possible to evaluate. 

\section{Deep Learning}
Until recently the field of NLP were dominated by older machine learning
techniques utilising linear models trained over very high-dimensional and sparse
feature vectors. Recently the field has switched over to neural networks over
dense inputs instead\cite[p.~1 - 2]{goldberg2015primer}.

What all neural networks have in common is that they are trying to find a
functional relationship for the data, with the specific form of the function
depending on the task. For NMT this reduces to finding the function $f : \bm{x}
\in \lang{X} \to \bm{y} \in \lang{Y}$ such that this $f$ maximizes the
likelihood $P(\bm{x} | \bm{y})$. Indeed many of the neural networks in existence
has been shown to be universal approximators, theoretically being able to
simulate a big set of nice functions\cite{Hornik:1989:MFN:70405.70408}.

We will go through the three most common architectures that are used in this thesis.

\subsection{Multilayer Perceptron}
Feedforward Neural Networks as MLP's are also called in the literature are
functions formed by composition by layers. The original MLP can be defined in
terms of a recurrence relation such that if we have input vectors of the form $\bm{x} \in
\mathbb{R}^{d_{in}}$ and output vectors of the form $\bm{y} \in
\mathbb{R}^{d_{out}}$, then an MLP with $L$ layers have the functional form of
\begin{equation}
  \label{eq:MLP_output}
  f(\bm{x} | \bm{\theta}) = \sigma_L(\bm{W}_L \bm{z}_{L-1} + \bm{b}_{L})
\end{equation}
where for any $l \in \{2, \dots, L-1\}$
\begin{equation}
    \bm{z}_l = \sigma_l(\bm{W}_l \bm{z}_{l-1} + \bm{b}_l)
\end{equation}
and with the base case
\begin{equation}
  \label{eq:MLP_base_case}
  \bm{z}_1 = \sigma_1(\bm{W}_1 \bm{x} + \bm{b}_1)
\end{equation}.

$\bm{W}_l$ and $\bm{b}_l$ may be of any dimension as long as it is dimensionally consistent
with the input and output of the layer and conform to the original input and output
dimensions. In this case we have that the parameters of the network are all of
the biases and weights for the layers, $\bm{\theta} = \{(\bm{W}_l, \bm{b}_l)_{l
  = 1}^L\}$.

\subsection{Recurrent Neural Networks}
A recurrent neural network acts upon sequences to produce an output. It is a
special kind of technique called parameter tying where parameters are shared
over many time-steps.

\subsection{Convolutional Neural Networks}



\subsection{old stuff}
A ubiquitous classifier within statistics is logistic regression. Logistic
regression uses an input vector $\bm{x}$ in order to give importance scores in
form of probabilities to different classes $y \in \{c_1, \dots, c_k\}$. It gets
its name from the logistic function
\begin{equation}
  \label{eq:logistic_function}
  \sigma(a) = \frac{1}{1 + e^{a}}
\end{equation}
which together with an affine transformation $\bm{W} \bm{x}$ yields the layer
\begin{equation*}
  \sigma(\bm{W} \bm{x})
\end{equation*}
which transforms values from a feature space $\bm{X} \subset \mathbb{R}^m$ into
probabilities\cite{Bishop:2006}.

Deep learning builds upon this intuition by recursively applying transformations
and activation functions, functions which in some sense maps input on to
\textit{ON/OFF} states. These functions take their functionality from an
abstraction from how neurons function when firing with regards to input,
mirroring how artificial neural networks have taken inspiration from how the
brain operates in the past. On a very basic level, neural networks are
characterised by stacked layers of affine transformations followed by activation
functions, where the output of one layer serves as the input to the next layer.
The final layer outputs $\hat{y}$ where the form of $\hat{y}$ depends on the application.
The hope is that after training the model using
backpropagation\cite{Rumelhart:1995:BBT:201784.201785} that the model is able to
predict satisfactory and drive down the specified loss.

Deep models are very powerful in that they are able to model complex functional
relationships. In our case we are looking at Supervised and Semi-supervised
learning, trying to find the relationship between $\bm{x} \in bm{X}$ and $\bm{y}
\in \bm{Y}$ of some kind of functional form $f(\bm{x}) \approx \bm{y}$.

Besides from the straightforward models where we stack logistic regressors
serially, neural networks have extended well beyond this into an extremely
diverse set of models that can capture different aspects of data such as long
term-dependencies through the architecture of Recurrent Neural Networks and
invariances by using convolutions. Many of these models have also found use in
NLP, especially in the form of RNN's which are well-suited for handling
language due to how it enables information to flow through
time\cite{graves_generating_2013}\cite{cho_learning_2014} and more recently
CNN's for finding representation over many different
scales\cite{semeniuta_hybrid_2017}\cite{yang_improved_2017}\cite{gehring_convolutional_2016}.

In a Bayesian setting each graphical model codifies how different random variables relate
to each other in terms of independency. This is specified by the Directed
Acyclic Graph where each arrow signifies a conditional relationship between
$\bm{x}$ and $\bm{y}$. A full description of how graphical models ,

\section{Natural Language Processing}

Humans use natural language everyday to convey concepts and abstractions to each
other in an efficient manner. However, compared to formal languages found in
mathematics and programming languages, the natural languages we use are often
ambiguous systems filled with rules and exceptions\cite{Rosenfeld00twodecades, sep-computational-linguistics}.

Natural Language Processing (Hereafter NLP) is and old field that for a long
time developed in parallel with the field of machine learning and
computational statistics which deals with how to process information coming from
human languages and is split up into several subfields such as Machine
Translation, statistical parsing and sentiment analysis\cite{sep-computational-linguistics}.

\subsection{Language model}
If we define a sentence to be a tuple of words $\bm{w} = (w_1, \dots,
w_l)$ such that each word is an atomic element $w_i \in \mathbf{D}$, where $\mathbf{D}$ is the
dictionary of words in our language, then the joint distribution of a word with
respect to the underlying probability measure can be rolled out using the
probabilistic chain rule which is just repeated application of the original
product rule\ref{eq:product_rule}
\begin{equation}
  \label{eq:conditional_language_probability}
  P(\bm{w}) = \prod_{t = 1}^TP(w_t | w_1, \dots, w_{t-1})
\end{equation}
where $w_t$ is the $t$'th word.\cite{Bengio:2003:NPL:944919.944966}

From this starting point analysis and generation of other language models can start.

\subsection{Word embeddings}
Breaking down sentences at a word level and processing them into a form that
encodes information efficiently is a problem which has gained notorious
recognition, leading to algorithms such as word2vec and
Glove\cite{DBLP:journals/corr/abs-1301-3781, Pennington14glove:global,
  Mikolov:2013:DRW:2999792.2999959}. However, these techniques work less well in
a neural network setting where instead finding the best embedding jointly while
optimizing the model is preferred\cite[p.~5-7]{goldberg2015primer}.

A straightforward way to represent the various words of the dictionary is as
one-hot-encoded vectors such that a word $w \in \mathbf{D}$, such that size of
$\mathbf{D}$ is $D$ with an index $i$
given by its place in the dictionary sorted alphabetically in descending order
will have the vector representation
\begin{equation}
  \label{eq:one_hot_encoding}
  \text{one-hot}(w) =
  \begin{bmatrix}
    0 \\
    \vdots \\
    0 \\
    1 \\
    0 \\
    \vdots \\
    0
  \end{bmatrix}
\end{equation}\cite[p.~6]{goldberg2015primer}
such that $\text{one-hot}(w)_{j} = \delta_{ij}$.

While this is a conceptually easy to understand, it fails to account for the
curse of dimensionality as the size of the vocabulary might grow to millions of
entries and the fact that the cosine similarity of two words $v_1, v_2 \in
\mathbf{D}$ is zero unless they are the same word
\begin{equation}
  \label{eq:cosine_similarity}
  \cos_{similarity}(\text{one-hot}(v_1), \text{one-hot}(v_2)) = \delta_{v_1 v_2}
\end{equation}. This means that no meaning is embedded in the vector space
except for the location in the sorted dictionary. Instead we would like to
associate each word in the vocabulary with a distributed \textit{word feature
  vector}, a dense, real-valued vector in $\mathbb{R}^m$; express the joint probability
function \ref{eq:conditional_language_probability} of word sequences in terms of
the feature vectors of these words in the sequence and simultaneously learn the
word feature vectors and the parameters of the model which dictates the form of
the probability function; $\bm{\theta}$. After this is done words which share
similarities in some sense such as \texttt{Dog, Puppy} would have a higher
similarity score than unrelated concepts such as \texttt{Dog, Bulwark}\cite{Bengio:2003:NPL:944919.944966}.

We may represent this in a mathematical form by trying to find a mapping $C$
from any element $w \in \mathbf{D}$ such that $C(w) \in \mathbb{R}^m$, in our
case $C$ is a linear mapping, using the one-hot encoded form of $\bm{w} =
\text{one-hot}(w)$ as the
initial representation of the words. This also means that we may represent $C$
as a matrix, $\bm{C} \in \mathbf{R}^{m \times D}$, thus the word feature vector of
the learned embedding can be represented by the matrix multiplication
$\bm{C}\bm{w}$.

\subsection{Neural machine translation}
For a long time the dominant paradigm within machine translation was to use
phrase based machine translation systems\cite{Koehn:2003:SPT:1073445.1073462, Koehn:2007:MOS:1557769.1557821},
however since a couple of year back, modelling the word or character level
directly with neural networks, so called NMT has become the best performing
method\cite{wolk_neural-based_2015, wu_googles_2016}.

Most NMT models work in terms of an encoder-decoder architecture where the
encoder extracts a fixed length representation $\bm{c}$, often called a context
vector, from a variable length input sentence $\bm{x} \in \lang{X}$, and the
decoder uses this representation to generate a correct translation $\bm{y} \in
\lang{Y}$ from this representation\cite{cho_properties_2014}.

\begin{figure}[H]
  \includestandalone[width=\textwidth]{./scripts/tikz_code/encoder_decoder}% 
%  \includegraphics[width=\textwidth]{encoderdecoder.pdf}
    \caption{Encoder decoder schematic}
  \label{fig:encoder_decoder}
\end{figure}

\section{Optimization}

\subsection{Problem formulation}

Most of machine learning may be recasted as an optimization problem. In a fully
probabilistic setting, the function to optimise is simply the joint probability
of the data, parametrised by the model. In a likelihood setting this reduces to
finding the maximizer of equations of the form \ref{eq:log-likelihood}.

Numerical optimization has been applied to many parts of science and thus many
different algorithms have been developed to tackle different problems. However,
most of the theoretical results that exist deal with convex optimization, where
the function we are trying to optimize is particularly well-behaved, leading to
theoretical guarantees on the solution converged to. The surface of the function
we are trying to optimize in machine learning in general and deep learning in
particular is not well-behaved, being highly non-linear and non-convex, meaning
most theoretical results that exist do not apply to this domain\cite{choromanska_loss_2014}.

Nevertheless, while theoretical results are lacking, there has been substantial
advances in various optimisation techniques fit to attack the highly non-linear,
non-convex and high-dimensional optimization problems of learning in deep
models, particularly from the Stochastic Gradient Descent. This has led to
numerous gradient descent-like algorithms used machine learning\cite{Ruder17}.

\subsection{Stochastic Gradient Descent}

For a normal probabilistic machine learning problem, we have data $\mathcal{D}$
that we try to model with a model $\mathcal{M}$ parametrised by parameters
$\bm{\theta} \in \Theta$. The optimization problem can in the most general case be recast
as an effort to find the parameters $\bm{\theta}_{ML}$ that maximizes the
log-likelihood function \ref{eq:log-likelihood}.

As this can't be found analytically we have to resort to numerical schemes to
find good candidates which hopefully will be close to the true solution
$\bm{\theta}_{ML}$. The first candidate is called Gradient Descent (GD).
GD approximates the gradient as if it was linear and takes steps to maximize the
probability through a hill-climbing scheme. If we let $\mathcal{D} =
\{\bm{z}_i\}_{i = 1}^n$ such that $\bm{z_i}$ is any general datapoint, $Q$ an
objective function that we try to maximize, such that $Q : \Theta \times
\mathcal{Z} \to \mathbb{R}$, then the update using GD looks as follows
\begin{equation}
  \label{eq:GD_update}
  \bm{\theta}_{t + 1} = \bm{\theta}_t - \gamma_t \frac{1}{n} \sum_{i = 1}^n \nabla_{\bm{\theta}} Q(z_i, \bm{\theta}_t)
\end{equation}.
While $\gamma_t$ may vary with $t$, it is often fixed for practical reasons. As
we can see GD takes into account all of the datapoints in the set, in some sense
updating the parameters with respect to the true linear approximation of the gradient.

However, with the huge size of data existing today, it is often not feasible to
calculate the update for all datapoints in the dataset due to the time it would
take and memory it would take to store the gradients.

Stochastic Gradient Descent is a similar algorithm to GD that instead of
calculating the gradient with respect to the whole dataset calculates an
approximate gradient, hence the word \textit{stochastic}, as it picks a random
subset of the dataset to update $\bm{\theta}$ with regards to. If we let $I_t$
be a random subset of the indices of $\mathcal{D}$ of size $m$ then SGD does the
following update
\begin{equation}
  \label{eq:SGD_update}
  \bm{\theta}_{t + 1} = \bm{\theta}_t - \gamma_t \frac{1}{m} \sum_{i \in I_t} \nabla_{\bm{\theta}} Q(z_i, \bm{\theta}_t)
\end{equation}\cite{series/lncs/Bottou12}\cite[p.~240]{Bishop:2006}.

The stochasticity has been shown to act as a regularizer and several analyses of
this in terms of a Bayesian framework has been done, explaining the stationary
behaviour of SGD with constant learning rate after
convergence\cite{mandt_stochastic_2017, mandt_variational_2016}

This solves the problem of memory and time, but also improves on GD in terms of
acting as a sort of regulariser for the model if $\gamma_t$ is kept fixed.
Experimental results also show that SGD yields extremely good practical
convergence on many tasks, and was often the first algorithm to be tried on deep
learning problems before other algorithms improved on the issue of picking a
good learning rate.

\subsection{ADAM}
SGD has found widespread use within the machine learning community due to strong
experimental results and ease of use, especially in deep learning. Lately
though, a number of alternatives have sprung up that aims to improve the vanilla
SGD, such as RMSProp\cite{Tieleman2012} and AdaGrad\cite{Duchi:EECS-2010-24}.

Adam takes inspiration from RMSProp and AdaGrad. Technically, Adam keeps an
exponential running average of the first and second order statistics of the
gradient, using these to calculate an adaptive learning rate.

As laid out in the original paper, ADAM operates on a stochastic objective
function $f(\bm{\theta})$, which in our case would be the probability averaged
over our input batch. $f_t(\bm{\theta})$ denotes this function over different
timesteps, equally $g_t = \nabla_{\bm{\theta}} f_t(\bm{\theta})$. $m_t$ and
$v_t$ denotes the first and second order moments of the gradients, however,
these are biased and are corrected in the algorithm.

The whole algorithm is as follows.

\begin{algorithm}
  \caption{ADAM}\label{ADAM}
  \begin{algorithmic}[1]
    \Require $\alpha$: Stepsize
    \Require $\beta_1, \beta_2 \in [0, 1)$: Exponential decay rates of the moment estimates
    \Require $f(\bm{\theta})$: Stochastic objective function with parameters
    $\bm{\theta}$
    \Require $\bm{\theta}_0$: Initial parameter vector
    \State $m_0 \gets 0$ (Initialize $1^{\text{st}}$ moment vector)
    \State $v_0 \gets 0$ (Initialize $2^{\text{nd}}$ moment vector)
    \State $t \gets 0$ (Initialize timestep)
    \While{$\bm{\theta}_t$ not converged}
    \State $t \gets t + 1$
    \State $g_t \gets \nabla_{\bm{\theta}}f_t(\bm{\theta}_{t-1})$ (get gradients w.r.t stochastic objective at timestep $t$)
    \State $m_t \gets \beta_1 \cdot m_{t-1} + (1 - \beta_1) \cdot g_t$ (Update biased first moment estimate)
    \State $v_t \gets \beta_2 \cdot v_{t-1} + (1 - \beta_2) \cdot g^2_t$ (Update biased second raw moment estimate)
    \State $\hat{m}_1 \gets m_t / (1 - \beta^t_1)$ (Compute bias-corrected first moment estimate)
    \State $\hat{v}_1 \gets v_t / (1 - \beta^t_2)$ (Compute bias-corrected second raw moment estimate)
    \State $\bm{\theta}_t \gets \bm{\theta}_{t-1} - \alpha \cdot \hat{m}_i/(\sqrt{\hat{v}_t} + \epsilon)$ (Update parameters)
    \EndWhile
    \Return $\bm{\theta}_t$ (Resulting parameters)
  \end{algorithmic}
\end{algorithm}

Experimentally Adam has shown very good results on training various deep
learning models such as MLP's, CNN's and RNN's so we will use it to train our
models throughout this dissertation\cite{kingma_adam:_2014}.