\chapter{Introduction}
\label{IntroductionCh}

% Layout of introduction

%%%
%%% Introduce the field briefly: NLP, VAE, Generative models for language
%%%

Natural language processing (NLP) is an old field with roots in many different
disciplines including but not exclusive to electrical engineering, artificial
intelligence and linguistics \cite[p.~10-15]{Jurafsky:2000:SLP:555733}. Machine
learning often view NLP as a statistical problem. A
language model is a statistical model that trained on a training corpus assign
probabilities to new sentences \cite{Bengio:2003:NPL:944919.944966}. Latent variable models enables these
language models to generate novel sentences from an underlying stochastic
variable. While using probabilistic models enables us to do inference in a
principled manner, it is non-trivial to optimise the log-likelihood
specified by a latent variable model. Variational Inference bounds the log-likelihood
from below using the Evidence Lower Bound (ELBO) by
introducing an approximate distribution, $q(\bm{z})$, of the conditional probability of the
latent variable, $p(\bm{z} | \bm{x})$ \cite{blei_variational_2017}.

%%%
%%% Recent advances: ELBO and relation to SGVB, Relation to NLP, training strong
%%% generative models
%%%

Although Variational Inference specifies ways of approximating the
log-likelihood by a variational distribution $q$ it does not specify the form of
this $q$. Variational Autoencoder (VAE) \cite{kingma_auto-encoding_2013}, also
under the name of stochastic backpropagation \cite{2014arXiv1401.4082J}, is a
framework for training deep generative models with latent variables by
introducing a recognition model $q_{\bm{\varphi}}(\bm{z} | \bm{x})$. Except for the most basic of
models the ELBO function is not tractable. However, it is possibly to
approximate the ELBO by sampling the latent variable instead of
integrating it out. Using the reparameterization trick to sample this latent
variable gives us the differentiable unbiased Stochastic Gradient Variational
Bayes (SGVB) estimator of the ELBO, leading to the Auto-Encoding Variational
Bayes (AEVB) algorithm. AEVB optimises the introduced recognition model
$q_{\bm{\varphi}}(\bm{z} | \bm{x})$ jointly with the generative model
$p_{\bm{\theta}}(\bm{z}, \bm{x})$ with respect to the SGVB objective
\cite{kingma_auto-encoding_2013}. As the recognition model maps the input sentences to the latent space, it effectively tries to compress
information about the sentences through $\bm{z}$, resulting in distributed
latent representations of sentences \cite{bowman_generating_2015}.

%%%
%%% What this thesis tries to do
%%%

In this thesis we introduce a deep generative model which generates sentences in two
different languages, $\lang{X}$ (EN) and $\lang{Y}$ (FR), from one latent
variable, $\bm{z}$. The generative model is based on the WaveNet neural network
\cite{DBLP:journals/corr/OordDZSVGKSK16} and is trained using the AEVB
algorithm. We show how different recognition models (MLP, RNN, WaveNet) affect
the final performance and training time in the mono-language generative model and based on this
performance we use the MLP recognition model for the twin-language generative
model. Letting the generative WaveNet model in the twin-language model differ in
depth and power, we evaluate how this affects the latent representation.

Learning the recognition model $q_{\bm{\varphi}}(\bm{z} | \bm{x}, \bm{y})$
theoretically lets us do translation. By omitting the dependency of one of the
languages, here $\lang{Y}$, we force $q_{\bm{\varphi}}(\bm{z} | \bm{x}, \bm{y})
\propto q_{\bm{\varphi}}(\bm{z} | \bm{x})$ giving us a way to perform mappings of $\lang{X} \to
\lang{Y}$. We use metrics including the SGVB estimate of the ELBO
and an upper bound on the perplexity to evaluate the models on language
generation, reconstruction and translation. We show that the twin-language model
avoids KL-collapse for all models and performs well on reconstruction and
generation, but fails to translate sentences.

%%%
%%% Layout of the thesis
%%%

The thesis follows the following structure. In Chapter 2 we review the necessary
background knowledge to understand the theory and experiments conducted. Chapter
3 lays out how variational inference and how the AEVB algorithm works,
and also specifies the mono and twin-language latent variable model. Chapter 4 presents the
experimental setup and a discussion of results and chapter 5 the conclusion and future work. The
appendix shows an excerpt of generated sentences for all the different models
together with plots.
