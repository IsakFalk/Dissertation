\chapter{Introduction}
\label{IntroductionCh}

Natural Language Processing, hereafter called NLP, is a subfield within
artificial intelligence almost 
as old as the field itself. Briefly, NLP can be defined as the study of the
properties of natural language and how these properties may be used
to answer questions that humans who master the language can do naturally.
Clearly, if we are to enable machines to cooperate with human beings, it is of
utmost importance that they can speak our language, since we are not very apt in
speaking theirs!

NLP has a plethora of different subfields, however, in this thesis we will limit
ourselves to the field of machine translation. Machine translation has long been
a cornerstone of NLP and has undergone many different guises from the beginning
of the 1940's until today.

Machine translation can essentially be seen as a problem of learning a model
that maps from one language $\lang{X}$ to another language $\lang{Y}$. Formally,
we have a dataset $\{(\bm{x}_i, \bm{y}_i)\}_{i = 1}^{N}$, such that $\bm{x}_i
\in \lang{X}, \bm{y}_i \in \lang{Y}$ represent the same sentence in two
different languages. The goal is then to find a mapping $f : \lang{X} \to
\lang{Y}$ such that $f(\bm{x}) \approx \bm{y}$ with regards to some chosen
metric.

From this several different ways of looking at language has emerged. The two
different ways depends on how granular you are going. For latin based languages
and other languages with a phonetic writing system, this means deciding if the
atomic symbols should be characters of words. Character level gives the
advantage that your dictionary of characters i finite, it consists of your
alphabet. Since it's finite there's less of a worry of new characters being
introduced. The data also gets bigger. For words you have a problem that the
dictionary swells, often being bigger than 100000 unique words, which are
troublesome when it comes to calculate the normalizer of the softmax in order to
get a distribution of the output from the raw model.

From the chain rule of probability we have that any sentence $\bm{x}$ is such
that the joint probability of the atomic units may be rewritten in a recursive
form,
\begin{align*}
  p(\bm{x}) & = p(x_1, \dots, x_n) \\
            & = \prod_{i = 1}^np(x_i | x_{i - 1}, \dots, x_1)
\end{align*},
which shows that one way of modelling language is to try to capture this
temporal relationship using models catered for long term dependencies. The main
problem is one of scale. This can be seen from the Markovian models called
\textit{N-grams} which simply assumes that the dependency in languages can be
explained by $N$-dimensional tables, meaning that $N$-grams can be seen as
$N$-step markov models applied to language data. In practice, this means that we
assume the relationship
\begin{equation*}
  p(x_i | x_{i - 1}, \dots, x_1) = p(x_i | x_{i - 1}, \dots, x_{i - N})
\end{equation*}.

Although this is a reasonable assumption from a modelling point of view, given
that we take $N$ to be large enough, it doesn't work well in practice. Due to
sparsity, these $N$-dimensional tables will be mostly filled with zeros. This is
simple to see whether working on a word or character level, I will just give an
example using words for maximum impact:

Assume we are trying to model the english language using a Tri-gram model,
meaning we are only considering grouped word of 3. Putting together random words
from the dictionary of words in the English language gives us by a huge margin
gibberish, in terms of both semantic meaning and of grammar:
\begin{center}
  \begin{tabular}{ |c| } 
    \hline
    \textit{moucharaby epithelium sonlike}\\
    \textit{Bacchides lulliloo oneiromancer}\\
    \textit{actinology dihydroxy nonmineralogical}\\ 
    \textit{Homalonotus Vened dyspepsy}\\
    \textit{uncessant twee femorofibular}\\
    \hline
  \end{tabular}
\end{center}
. This is due to the fact that only extremely few tuples of word triplets are
actually valid in the sense that they can be said to exist naturally in the
English language. Mathematically, this means that the tables we get from running
maximum likelihood on these tables to find the actual probabilities, which is
just a matter of counting the number of times the triplets occur compared to the
number of times that the starting symbol occur in the corpus we have at hand.

Neural Machine Translation, hereafter called NMT, is the use of Neural Networks
as the models inside 
the machine translation systems. NMT can be trained end-to-end by specifying the
data, architecture and the various other components that make up the model
specification. While NMT is very data-hungry, mostly getting its power from
being able to unearth the various rules and constructs in a language
(semantically and grammatically) through the use of the statistical information
existing within it, it is extremely well-suited for learning these rules given
enough data. This black box approach means that people without any knowledge of
language $\lang{X}$ and $\lang{Y}$ can train sophisticated translation systems
on par with state of the art results, solely relying on the data to speak for
itself.

In this thesis we will explore fully probabilistic models, meaning that any
statement about output languages can be given a probability score. Using the
language of probability enables use to make statements about the plausibility of
sentences and logical statements about these sentences using the laws of
probability. Practically, it means that we get a model which is generative, that
is, we can sample random variables of the hidden variable that encodes the
sentences in a language-agnostic which through the models can map back to the
sentences in the original languages. This gives us some hope that the model have
some kind of internal language model and not only learns the specific output
relation when mapping from $\lang{X}$ to $\lang{Y}$.

In essence, we will use recent techniques that enables us to train
latent-variable models, that is models where the observed output, in our case
the language sentence tuples depend on a hidden variable $\bm{z}$ that encodes
the information about the sentence in a language-agnostic way. Consider the
sentence \textit{The quick brown fox jumps over the lazy dog}. The sentence is
written using the English language, but it's easy to imagine the if we disregard
the language we are saying it in, whether it be German, \textit{Der schnelle
  braune Fuchs sprang über den faulen Hund} or in Latin, \textit{Lorem ipsum
  vulpes salit super piger canis}, there is some underlying meaning which all
languages are trying to convey. Our model tries to encode this meaning in terms
of a hidden stochastic variable $\bm{z}$.

This leads to the realisation given that we can encode $\bm{z}$ properly we
should be able to translate from language $\lang{X}$ to $\lang{Q}$ without the
model having ever seen a sentence pair of the form $(\bm{x}, \bm{q}), \bm{x} \in
\lang{X}, \bm{q} \in \lang{Q}$.

\section{Things to talk about}

\begin{itemize}
\item NLP (History, challenges)
\item Deep Learning (What it is)
\item Neural Machine Translation
\end{itemize}









